\documentclass[uplatex,a4j]{jsreport}
\usepackage{thesis}

\begin{document}
\chapter{準備}
\label{準備}
% 読んでもらうことに必要なこと
% 自然言語処理のこと,必要な言葉
%html5のtokenizer仕様のことは3章
% \section{節, 句, 品詞タグ}
% % 本論文ではいくつかの品詞タグを使用する. 
% 節は主語と述語がある文のことで, 句は2つ以上の単語によって構成されている単語の集まりである. (本論文では節と句は特に区別しない.) 
% 本論文で使用する, いくつかの主要な節,句,単語のタグ名とその意味~\cite{pennTreebankTags}について書く. 

% 節, 句の種類
% \begin{alignat*}{3}%&{\rm 節, 句の種類 } && && \\
%      &{\rm S } &\quad &{\rm : 文 }&\quad& \\
%      &{\rm VP }& &{\rm : 動詞句 }& & \\
%      &{\rm NP }& &{\rm : 名詞句 }& & \\
%      &{\rm PP }& &{\rm : 前置詞句 }& & \\
% \end{alignat*}

% 単語の種類(品詞)
% \begin{alignat*}{3}
%      &{\rm VB }&\quad &{\rm : 動詞 }& & \\
%      &{\rm NN }& &{\rm : 名詞} & & \\
%      &{\rm IN }& &{\rm : 前置詞, 従属接続詞} & & \\
%      &{\rm DT }& &{\rm : 限定詞} & & \\
%      &{\rm CC }& &{\rm : 等位接続詞} & & \\
%      &{\rm PRP }& &{\rm : 人称代名詞} & & \\
% \end{alignat*}

\section{自然言語処理}
自然言語は, 人間が同士が互いにコミュニケーションをとるために発展してきた言語である. そして自然言語をコンピュータで処理する技術を自然言語処理(Natural Language Processing)と呼んでいる. 
自然言語処理をすることによって, 自然言語の文章を構成している単語の解析(単語の品詞や、原型の取得など)や文章の構造の解析, 意味の解析をすることが出来る. 
このセクションでは, 自然言語処理によって得られるそれらの情報に関しての説明をしていく. 
% 本論文で使用する自然言語処理に関して,
% % 使用ライブラリは 序論OR自然言語の章に移してもよいかも
% \subsection{使用ライブラリ}
% 実装では, 自然言語処理のライブラリとして, スタンフォード大学によって提供されている Stanford CoreNLP~\cite{manning-EtAl:2014:P14-5}を使用した.
% % Stanford CoreNLPは自然言語処理ツールのひとつであり,スタンフォード大学によって提供されている.
% % StanfordCoreNLPでは,
% % % ライブラリでどういう情報を得られるか
% % 形態素解析(品詞タグ付け、単語の原型の取得),構文解析,意味解析などが出来る.\\
% %以下の例で詳しく述べる.\\
% % pipelineの説明をする===
% (pipelineの説明をする)\\
% %====

% ``Mika likes her dog's name.''をStanford CoreNLPで自然言語処理をさせる.
\subsection{トークン分割, 品詞タグ付け, レンマ化}
% 概要
トークン化では, 文章を単語という単位に分割する操作である. 
そして品詞タグ付けは, 文章をトークン化した後, その単語の品詞を調べる操作である. 
例えば, 品詞には以下のような種類があり, 単語ごとに品詞が定められる. 
% 単語の種類(品詞)
\begin{alignat*}{3}
     &{\rm VB }&\quad &{\rm : 動詞 }& & \\
     &{\rm NN }& &{\rm : 名詞} & & \\
     &{\rm IN }& &{\rm : 前置詞, 従属接続詞} & & \\
     &{\rm DT }& &{\rm : 限定詞} & & \\
     &{\rm CC }& &{\rm : 等位接続詞} & & \\
     &{\rm PRP }& &{\rm : 人称代名詞} & & \\
\end{alignat*}
そして, レンマ化とはその単語の原型を調べる操作である. この操作により, 例えば三人称単数現在形の動詞の原型を調べられる. 

例1
\begin{lstlisting}[basicstyle=\ttfamily\footnotesize, frame=single][htbp]
     Mika likes her dog's name.
\end{lstlisting}
これをトークン分割, 品詞タグ付け及びレンマ化をすると, \\
単語/品詞タグ : 
Mika/NNP $\hspace{10pt}$ likes/VBZ $\hspace{10pt}$ her/PRP\$ $\hspace{10pt}$ dog/NN $\hspace{10pt}$ 's/POS $\hspace{10pt}$ name/NN $\hspace{10pt}$ ./.\\
単語/原型 : 
Mika/Mika $\hspace{10pt}$  likes/like $\hspace{10pt}$ her/she $\hspace{10pt}$ dog/dog $\hspace{10pt}$ 's/'s $\hspace{10pt}$ name/name $\hspace{10pt}$ ./.\\
となる.
\subsection{構文木解析}
自然言語の文章には, 節や句などのまとまりが存在する. 
節は主語と述語がある文のことで, 句は2つ以上の単語によって構成されている単語の集まりである. (本論文では節と句は特に区別しない.) 
例えば, 節や句の種類には以下のようなものがある. 
\begin{alignat*}{3}%&{\rm 節, 句の種類 } && && \\
     &{\rm S } &\quad &{\rm : 文 }&\quad& \\
     &{\rm VP }& &{\rm : 動詞句 }& & \\
     &{\rm NP }& &{\rm : 名詞句 }& & \\
     &{\rm PP }& &{\rm : 前置詞句 }& & \\
\end{alignat*}
% 概要
そして構文木解析とは, 単語を節や句という単位でグループ化し, その構成を木構造で表現するものである.
構文木解析をすることによって, 文章の構造や単語同士のまとまりを調べることが出来る.\\

例1の``Mika likes her dog's name.''を構文木解析すると.\\
\Tree [.S [.NP [.NNP Mika ] ]
           [.VP
              [.VBZ likes ]
              [.NP [.NP [.PRP\$ her ][.NN dog ][.POS 's ] ]
                    [.NN name ] ]
           ]
           [.. . ]
      ]\\
の様になる.
\subsection{係り受け解析(dependency parse)}
係り受け解析とは、節や単語間の関係を調べるものである. \\%を係り受けタグによって関連付け, の関係を解析するものである.\\
%%係り受けタグとは
目的語, 修飾語などといったの単語や節の関係性を表す係り受けタグを使用し, それらを関連付ける.%%使い, であり, 
例えば, 単語Bが単語Aの目的語である場合, 
A $\xrightarrow{目的語}$ B 
という風に添え字付きの矢印で表記する.\\

例の, ``Mika likes her dog's name.''を係り受け解析すると, 図\ref{dependency}の様になり, 
``Mika''が好きなのは, ``name'', 
``name''は ``dog'' の名前, 
``dog'' は 彼女(her)の犬である\\
という様に, 単語の目的語, 所有しているものを示してる単語が解析されている.\\
\begin{figure}[h]
     \centering
     \includegraphics[keepaspectratio, scale=0.7]
          {figure/dependencyEx.png}
     \caption{係り受け解析}
     \label{dependency}
   \end{figure}
%明示的に示されている.\\

\subsection{固有表現抽出(named entity recognition)}
固有表現抽出とは, 文章内に数字や時間、アドレス、人間、地名を意味する単語があった場合, 「この単語は地名を表すものである」という風に, 固有な表現を解析するものである.\\

例1の``Mika likes her dog's name.''の場合は, 
``Mika'' が``人間''を表す単語であるということが固有表現抽出によって出力される. 
\subsection{参照関係の解析(coreference)}
参照関係の解析とは,文章内で同じものを指し示している単語が複数ある時, それらを関連付けさせ, 抽出するものである. 
これによって, ``it''や``he''などの指示語の指し示すものを見つけることなどが出来る.\\

例1の``Mika likes her dog's name.''の場合は, 
``Mika'', ``her''が同一のものであることが参照関係の解析によって分かる.

\section{自然言語処理のライブラリの使用}
本研究では, 自然言語処理のライブラリとして, スタンフォード大学によって提供されている Stanford CoreNLPを使用した. 
その使用方法について説明をしていく. 
\subsection{a}
\end{document}