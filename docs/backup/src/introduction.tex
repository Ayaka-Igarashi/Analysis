\documentclass[uplatex,a4j]{jsreport}
\usepackage{thesis}

\begin{document}
\chapter{序論}

HTML5は, 現代社会にて広く普及しているWebページを作成するためのマークアップ言語のひとつである.
% html5の構文解析に関する研究の話 + どうしてやろうと思ったか
% 過去に行われてきたHTML5の構文解析の検証に関する研究には, 
過去に, HTML5の構文解析に関する研究として, 
XSS(クロスサイトスクリプティング)保護機構であるXSSAuditorのトランスデューサでのモデル化による有効性の検証~\cite{XSSAuditor} ~\cite{トランスデューサの包含関係}や, 
HTML5構文解析仕様の到達可能性の解析 ~\cite{HTML5Testing}などが行われていた. 
それらの研究では実装段階において, 自然言語によって記述されているHTML5の字句解析仕様から, 手作業でその命令, 動作を抽出している. 
そこで, 構文解析の検証における実装の負担を減らすために, 仕様書からの命令の抽出,  形式化を自動化したいと考えた. 
よって, 自然言語をコンピュータで処理する技術である, 自然言語処理を使い, HTML5の字句解析仕様から命令を抽出することを試みた.
% さらに, 仕様の形式化の自動化が出来るようになることによって, 仕様の変更が行われる際, 仕様の定式化を自動化していれば, 変更への対応が楽になるというメリットもある.
% 一般的な仕様の定式化が出来るツールがあると嬉しい.
% 自然言語は, 人間が同士が互いにコミュニケーションをとるために発展してきた言語である. そして自然言語をコンピュータで処理する技術を自然言語処理(Natural Language Processing)と呼んでいる.

%ARMの研究の話
仕様書から自然言語処理を用いて命令抽出を試みた研究としては, 
仕様書の文章の構造や自然言語処理による品詞タグの情報を使い, 命令を抽出する研究~\cite{7272551}や, 
機械語命令ARMを対象とした, 仕様書の意味論抽出行う研究~\cite{arm}が行われている. 
ARMを対象とした研究では, 自然言語処理による構文木解析などの結果を用いて, 意味論の抽出を行っていた.\\

%====何か書く==-
% (何か書く)

% 図\ref{流れ}がHTML5の字句解析仕様の意味解析の概要である.\\
% % 概要の図
% \begin{figure}[h]
%     \centering
%     \includegraphics[keepaspectratio, scale=0.6]
%          {figure/流れ2.png}
%     \caption{流れ(仮)}
%     \label{流れ}
% \end{figure}
% \\
% ここの部分を何章で説明する, とかを書く
本論文では, まず\ref{準備}章で自然言語処理の基礎知識を述べる.
次に\ref{字句解析仕様}章で HTML5 の字句解析器の主な仕様, 動作について述べる.
\ref{形式}章で抽出する命令の形式をBNFとして述べ, 
\ref{自然言語処理}章で自然言語処理ライブラリを用い, それをHTML5字句解析仕様に適用させ, 
\ref{命令抽出}章で自然言語処理の出力をもとに仕様書の命令の抽出を行った. 
\ref{実装}章で抽出した命令をもとに字句解析をするインタプリタを作成し, 
字句解析のテストデータを用い, 抽出した命令の正しさを検証した.
\end{document}