\documentclass[uplatex,a4j]{jsreport}
\usepackage{thesis}

\begin{document}
\chapter{序論}

自然言語は、人間が同士が互いにコミュニケーションをとるために発展してきた言語である.そして自然言語をコンピュータにで処理する技術を自然言語処理(Natural Language Processing)と呼んでいる.
本論文では自然言語処理の技術を使ってHTML5の字句解析仕様から命令を抽出することを試みた.

HTML5の構文解析の検証に関する研究には***がある. ~\cite{XSSAuditor}~\cite{トランスデューサの包含関係}~\cite{HTML5Testing}それらの研究は自然言語によって書かれている仕様を手作業で抽出しているので, 仕様の形式化を自動化したいと思った. 
仕様の形式化の自動化のメリットとしては, 仕様の変更が行われることがあるため, その時に仕様の定式化を自動化していると嬉しい.

% 一般的な仕様の定式化が出来るツールがあると嬉しい.

%ARMの研究の話
仕様書から自然言語処理を用いて命令抽出を試みた研究としては, 機械語命令ARMの仕様書を対象とした研究があり, 
自然言語処理の構文木解析の結果を用いて, 命令の抽出を行っていた.~\cite{arm}\\

% 使用ライブラリ
実装では, 自然言語処理のライブラリとして, スタンフォード大学によって提供されている Stanford CoreNLP~\cite{manning-EtAl:2014:P14-5}を使用した.


%====何か書く==-
(何か書く)

図\ref{流れ}がHTML5の字句解析仕様の意味解析の概要である.\\
% 概要の図
\begin{figure}[h]
    \centering
    \includegraphics[keepaspectratio, scale=0.6]
         {figure/流れ.png}
    \caption{流れ(仮)}
    \label{流れ}
\end{figure}
\\
% ここの部分を何章で説明する、とかを書く
本論文では, まず\ref{準備}章で自然言語処理の基礎知識を述べる.
次に\ref{字句解析仕様}章で HTML5 の字句解析器の主な仕様, 動作について述べる.
\ref{形式}章で抽出する命令の形式をBNFとして述べ, 
\ref{自然言語処理}章で自然言語処理ライブラリを用い, それをHTML5字句解析仕様に適用させ, 
\ref{命令抽出}章で自然言語処理の出力をもとに仕様書の命令の抽出を行った. 
\ref{実装}章で抽出した命令をもとに字句解析をするインタプリターを作成し, 
\ref{評価}章で字句解析のテストデータを用い, 抽出した命令の正しさを検証した.
\end{document}