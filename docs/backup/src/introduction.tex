\documentclass[uplatex,a4j]{jsreport}
\usepackage{thesis}

\begin{document}
\chapter{序論}

自然言語は, 人間が同士が互いにコミュニケーションをとるために発展してきた言語である.そして自然言語をコンピュータにで処理する技術を自然言語処理(Natural Language Processing)と呼んでいる.
本論文では自然言語処理の技術を使ってHTML5の字句解析仕様から命令を抽出することを試みた.

% html5の構文解析に関する研究の話 + どうしてやろうと思ったか
過去に行われてきたHTML5の構文解析の検証に関する研究には, XSS(クロスサイトスクリプティング)保護機構であるXSSAuditorのトランスデューサでのモデル化による有効性の検証~\cite{XSSAuditor} ~\cite{トランスデューサの包含関係}や, 
HTML5構文解析仕様の到達可能性の解析 ~\cite{HTML5Testing}などがある. 
それらの研究では実装の評価において, 自然言語によって書かれているHTML5字句解析仕様から, 命令を手作業で抽出している. 
よって構文解析の検証における実装の負担を減らすために, 仕様書からの命令の抽出,  形式化を自動化したいと考えた. 
さらに, 仕様の形式化の自動化が出来るようになることによって, 仕様の変更が行われる際, 仕様の定式化を自動化していれば, 変更への対応が楽になるというメリットもある.
% 仕様の変更が行われることがあるため, その時に仕様の定式化を自動化していると嬉しい.

% 一般的な仕様の定式化が出来るツールがあると嬉しい.

%ARMの研究の話
仕様書から自然言語処理を用いて命令抽出を試みた研究としては, 機械語命令ARMの仕様書を対象としたものがあり, 
そこでは, 自然言語処理の構文木解析などの結果を用いて, 命令の抽出を行っていた.~\cite{arm}\\

% 使用ライブラリ
実装では, 自然言語処理のライブラリとして, スタンフォード大学によって提供されている Stanford CoreNLP~\cite{manning-EtAl:2014:P14-5}を使用した.


%====何か書く==-
(何か書く)

図\ref{流れ}がHTML5の字句解析仕様の意味解析の概要である.\\
% 概要の図
\begin{figure}[h]
    \centering
    \includegraphics[keepaspectratio, scale=0.6]
         {figure/流れ2.png}
    \caption{流れ(仮)}
    \label{流れ}
\end{figure}
\\
% ここの部分を何章で説明する, とかを書く
本論文では, まず\ref{準備}章で自然言語処理の基礎知識を述べる.
次に\ref{字句解析仕様}章で HTML5 の字句解析器の主な仕様, 動作について述べる.
\ref{形式}章で抽出する命令の形式をBNFとして述べ, 
\ref{自然言語処理}章で自然言語処理ライブラリを用い, それをHTML5字句解析仕様に適用させ, 
\ref{命令抽出}章で自然言語処理の出力をもとに仕様書の命令の抽出を行った. 
\ref{実装}章で抽出した命令をもとに字句解析をするインタプリターを作成し, 
\ref{評価}章で字句解析のテストデータを用い, 抽出した命令の正しさを検証した.
\end{document}