\documentclass[uplatex,a4j]{jsreport}
\usepackage{thesis}

\begin{document}
\chapter{序論}

HTML5は, 現代社会にて広く普及しているWebページを作成するためのマークアップ言語のひとつである.
% HTML5パーサーの概要を書く
HTML5文章はその描画の処理として, まず字句解析器によって文章がTokenという単位に分解され, 構文解析器によってそのToken列からDOMツリーを作成し, 描画を行う.%に基づくDocumentオブジェクト

% html5の構文解析に関する研究の話 + どうしてやろうと思ったか 過去に行われてきたHTML5の構文解析の検証に関する研究には, 
HTML5の構文解析を対象とした研究として, 過去に
XSS(クロスサイトスクリプティング)保護機構であるXSSAuditorのトランスデューサでのモデル化による有効性の検証~\cite{XSSAuditor} ~\cite{トランスデューサの包含関係}や, 
HTML5構文解析仕様の到達可能性の解析 ~\cite{HTML5Testing}などが行われている. 
それらの研究では実装段階において, 自然言語によって記述されているHTML5の字句解析仕様から, 手作業でその命令, 動作を抽出している. 
そこで, 構文解析の検証における実装の負担を減らすために, 仕様書からの命令の抽出,  形式化を自動化したいと考えた. 
よって, 自然言語をコンピュータで処理できる, 自然言語処理の技術を用いて, HTML5の字句解析仕様から命令を抽出する方法を模索し, それを自動化することを考えた.
% さらに, 仕様の形式化の自動化が出来るようになることによって, 仕様の変更が行われる際, 仕様の定式化を自動化していれば, 変更への対応が楽になるというメリットもある.
% 一般的な仕様の定式化が出来るツールがあると嬉しい.
% 自然言語は, 人間が同士が互いにコミュニケーションをとるために発展してきた言語である. そして自然言語をコンピュータで処理する技術を自然言語処理(Natural Language Processing)と呼んでいる.

%ARMの研究の話
自然言語処理を用いて仕様書からの命令抽出を試みた研究としては, 
仕様書の文章の構造や自然言語処理による単語の品詞タグの情報を使い, 仕様書に記述されている命令を抽出する研究~\cite{7272551}や, 
機械語命令ARMを対象とした, その仕様書の意味論抽出行う研究~\cite{arm}が行われている. 
ARMを対象とした研究では, 自然言語処理による構文木解析などの結果を用いて, 意味論の抽出を行っていた.

% 何をやったか簡単に書く
本研究では, HTML5の字句解析仕様に対して, その命令の形式化をし, 仕様書の文章に自然言語処理を適用し, 構文解析や意味解析を行い, 形式化された命令を抽出する方法を考えた. 
その際, 仕様書の文章に対して直接自然言語処理を適用すると, 構文解析や意味解析の結果が適切なものが得られないことがあったので, 自然言語処理を適用する前の処理として, 特定の文字列の置き換えをするなどしたり, 
構文木解析の結果を用いて命令を抽出する際も, 条件分岐の文など、命令の抽出の仕方に関する工夫を行った. 
% 図\ref{流れ}がHTML5の字句解析仕様の意味解析の概要である.\\
% % 概要の図
% \begin{figure}[h]
%     \centering
%     \includegraphics[keepaspectratio, scale=0.6]
%          {figure/流れ2.png}
%     \caption{流れ(仮)}
%     \label{流れ}
% \end{figure}
% \\

% ここの部分を何章で説明する, とかを書く
本論文では, まず\ref{準備}章で自然言語処理に関する基礎知識を述べる.
次に\ref{字句解析仕様}章で HTML5 の字句解析器の仕様や具体的な動作について述べる.
そして\ref{形式}章で抽出する命令の形式化をし, 
\ref{自然言語処理}章でHTML5字句解析仕様への自然言語処理の適用の方法について述べ, 
\ref{命令抽出}章で自然言語処理の出力をもとにした, 仕様書の命令の抽出の方法について述べる. 
最後に, \ref{実装}章で抽出し形式化した命令をもとに字句解析をするインタプリタの実装について述べ, 
\ref{評価}章で字句解析インタプリタに対してテストを行い, 抽出した命令の正しさを検証した.
\end{document}