\documentclass[uplatex,a4j]{jsreport}
\usepackage{thesis}

\begin{document}
\chapter{抽出の形式}
\label{形式}
%% BNFを書く
\section{抽出する命令の形式}
以下のBNFの形式で仕様書から抽出する命令を定義する.\\
cList : CommandList $\cdots$ 命令文のリスト\\
c : Command $\cdots$ 命令文\\
b : Bool $\cdots$ 条件文\\
cval : CommandValue $\cdots$ 値\\
ival : InplementVariable $\cdots$ 代入される変数\\
% \begin{eqnarray*}
%     {\rm cList }&\bnfdef& c :: cList \bnfor Nil\\
% \end{eqnarray*}
\begin{eqnarray*}
  {\rm cList }&\bnfdef& \mbox{ c :: cList} \bnfor \mbox{ Nil}\\
  {\rm c }&::=& \mbox{ If(b, cList$_1$, cList$_2$) // if b then cList$_1$ else cList$_2$}\\
    &|& \mbox{ Ignore() // 何もしない} \\
    &|& \mbox{ Switch(cval) // 状態cvalへ遷移する} \\
    &|& \mbox{ Reconsume(cval) // 状態cvalへ遷移. この状態で消費した文字を,次の状態で再度消費する.} \\
    &|& \mbox{ Set(ival, cval) // ivalにcvalを代入する(ival $\leftarrow$ cval)} \\
    &|& \mbox{ AppendTo(cval, ival) // ivalにcvalを追加する(ival $\leftarrow$ ival + cval)} \\
    &|& \mbox{ Emit(cval) // トークンcvalを排出する} \\
    &|& \mbox{ Create(cval) // トークンcvalを新たに作る} \\
    &|& \mbox{ Consume(cval) // 文字cvalを消費する} \\
    &|& \mbox{ Error(string) // エラーstringを排出する} \\
    &|& \mbox{ FlushCodePoint() // 一時バッファの内容を排出する} \\
    &|& \mbox{ StartAttribute() // 現在のtagTokenに新しい属性を加える} \\
    &|& \mbox{ TreatAsAnythingElse() // AnythingElseの処理内容を実行する} \\
    &|& \mbox{ AddTo(cval, ival) //ival $\leftarrow$ ival + cval } \\
    &|& \mbox{ MultiplyBy(ival, cval) //ival $\leftarrow$ ival * cval} \\
\end{eqnarray*}
\begin{eqnarray*}
    {\rm b }&::=& \mbox{ And(b$_1$, b$_2$)}\\
      &|& \mbox{ Or(b$_1$, b$_2$)} \\
      &|& \mbox{ Not(b)} \\
      &|& \mbox{ CharacterReferenceConsumedAsAttributeVal}() \mbox{ // CharacterReferenceCodeが属性の値として消費されているか} \\
      &|& \mbox{ CurrentEndTagIsAppropriate}() \mbox{ //EndTagTokenが適切であるか } \\
      &|& \mbox{ IsEqual(cval$_1$, cval$_2$)} \\
\end{eqnarray*}
\begin{eqnarray*}
  {\rm cval }&::=& \mbox{ StateName(string) // 状態名string}\\
    &|& \mbox{ ReturnState // return state} \\
    &|& \mbox{ TemporaryBuffer // temporary buffer} \\
    &|& \mbox{ CharacterReferenceCode // character reference code} \\
    &|& \mbox{ NewStartTagToken // 新しいstart tag token} \\
    &|& \mbox{ NewEndTagToken // 新しいend tag token} \\
    &|& \mbox{ NewDOCTYPEToken // 新しいDOCTYPE token} \\
    &|& \mbox{ NewCommentToken // 新しいcomment token} \\
    &|& \mbox{ CurrentTagToken // 一番新しく作られたtag token} \\
    &|& \mbox{ CurrentDOCTYPEToken // 一番新しく作られたDOCTYPE token} \\
    &|& \mbox{ CurrentAttribute // 一番新しく作られたattribute} \\
    &|& \mbox{ CommentToken // 一番新しく作られたcomment token} \\
    &|& \mbox{ EndOfFileToken // end of fileトークン} \\
    &|& \mbox{ CharacterToken(char) // character token : char} \\
    &|& \mbox{ LowerCase(cval) // cvalの小文字} \\
    &|& \mbox{ NumericVersion(cval) // 16進数表記されているcvalの数字としての値} \\
    &|& \mbox{ CurrentInputCharacter // 現在消費した文字} \\
    &|& \mbox{ NextInputCharacter // 入力文字列の一番最初の文字} \\
    &|& \mbox{ Variable(string) // 変数string} \\
    &|& \mbox{ CChar(char) // Char型の値char} \\
    &|& \mbox{ CString(string) // String型の値string} \\
    &|& \mbox{ CInt(int) // Int型の値int} \\
    &|& \mbox{ CBool(boolean) // Boolean型の値boolean} \\
\end{eqnarray*}
\begin{eqnarray*}
  {\rm ival }&::=& \mbox{ IReturnState // return state}\\
    &|& \mbox{ ITemporaryBuffer // 一時バッファ} \\
    &|& \mbox{ ICharacterReferenceCode // character reference code} \\
    &|& \mbox{ ICurrentTagToken // 一番新しく作られたtag token} \\
    &|& \mbox{ ICurrentDOCTYPEToken // 一番新しく作られたDOCTYPE token} \\
    &|& \mbox{ ICurrentAttribute // 一番新しく作られたattribute} \\
    &|& \mbox{ ICommentToken // 一番新しく作られたcomment token} \\
    &|& \mbox{ IVariable(string) // 変数string} \\
    &|& \mbox{ INameOf(ival) // ivalの名前} \\
    &|& \mbox{ IValueOf(ival) // ivalの値} \\
    &|& \mbox{ IFlagOf(ival) // ivalのflag} \\
    &|& \mbox{ SystemIdentifierOf(ival) // ivalのsystem identifier} \\
    &|& \mbox{ PublicIdentifierOf(ival) // ivalのpublic identifier} \\
\end{eqnarray*}

string,char,int,booleanはそれぞれScalaの標準の型(String,Char,Int,Boolean)の値
\section{例}
Switch to the Data_state.\\
$\Rightarrow$ Switch(StateName(Data_state))\\

Append the lowercase version of the current input character to the current tag token's tag name.\\
$\Rightarrow$ Append(LowerCase(CurrentInputCharacter), INameOf(CurrentTagToken))\\
% \begin{eqnarray*}
%     {\rm c }&::=& \mbox{if $\langle$Bool$\rangle$ then $\langle$cList$\rangle_1$ else $\langle$cList$\rangle_2$}\\
%       &|& \mbox{ Ignore() // 何もしない} \\
%       &|& \mbox{ Switch($\langle$CommandValue$\rangle$) // 状態cvalへ遷移する} \\
%       &|& \mbox{ Reconsume($\langle$CommandValue$\rangle$) // 状態cvalへ遷移. この状態で消費した文字を,次の状態で再度消費する.} \\
%       &|& \mbox{ Set(<ImplementVariable>, $\langle$CommandValue$\rangle$) // ivalにcvalを代入する(ival $\leftarrow$ cval)} \\
%       &|& \mbox{ AppendTo($\langle$CommandValue$\rangle$, <ImplementVariable>) // ivalにcvalを追加する(ival $\leftarrow$ ival + cval)} \\
%       &|& \mbox{ Emit($\langle$CommandValue$\rangle$) // トークンcvalを排出する} \\
%       &|& \mbox{ Create($\langle$CommandValue$\rangle$) // トークンcvalを新たに作る} \\
%       &|& \mbox{ Consume($\langle$CommandValue$\rangle$) // 文字cvalを消費する} \\
%       &|& \mbox{ Error(string) // エラーstringを排出する} \\
%       &|& \mbox{ FlushCodePoint() // 一時バッファの内容を排出する} \\
%       &|& \mbox{ StartAttribute() // 現在のtagTokenに新しい属性を加える} \\
%       &|& \mbox{ TreatAsAnythingElse() // AnythingElseの処理内容を実行する} \\
%       &|& \mbox{ AddTo($\langle$CommandValue$\rangle$, $\langle$ImplementVariable$\rangle$) 
%                 //$\langle$ImplementVariable$\rangle = \langle$ImplementVariable$\rangle + \langle$CommandValue$\rangle$ } \\
%       &|& \mbox{ MultiplyBy(<ImplementVariable>, $\langle$CommandValue$\rangle$)} \\
% \end{eqnarray*}
% \begin{eqnarray*}
%     {\rm <Bool> }&::=& And(<Bool>, <Bool>)\\
%       &|& Or(<Bool>, <Bool>) \\
%       &|& Not(<Bool>) \\
%       &|& CharacterReferenceConsumedAsAttributeVal() \\
%       &|& CurrentEndTagIsAppropriate() \\
%       &|& IsEqual(<CommandValue>, <CommandValue>) \\
% \end{eqnarray*}
% \begin{eqnarray*}
%     {\rm <CommandValue> }&::=& StateName(<String>)\\
%       &|& ReturnState \\
%       &|& TemporaryBuffer \\
%       &|& CharacterReferenceCode \\
%       &|& NewStartTagToken \\
%       &|& NewEndTagToken \\
%       &|& NewDOCTYPEToken \\
%       &|& NewCommentToken \\
%       &|& CurrentTagToken \\
%       &|& CurrentDOCTYPEToken \\
%       &|& CurrentAttribute \\
%       &|& CommentToken \\
%       &|& EndOfFileToken \\
%       &|& CharacterToken(<Char>) \\
%       &|& LowerCase(<CommandValue>) \\
%       &|& NumericVersion(<CommandValue>) \\
%       &|& CurrentInputCharacter \\
%       &|& NextInputCharacter \\
%       &|& Variable(<String>) \\
%       &|& CChar(<Char>) \\
%       &|& CString(<String>) \\
%       &|& CInt(<Int>) \\
%       &|& CBool(<Boolean>) \\
% \end{eqnarray*}
% \begin{eqnarray*}
%     {\rm <ImplementVariable> }&::=& IReturnState\\
%       &|& ITemporaryBuffer \\
%       &|& ICharacterReferenceCode \\
%       &|& ICurrentTagToken \\
%       &|& ICurrentDOCTYPEToken \\
%       &|& ICurrentAttribute \\
%       &|& ICommentToken \\
%       &|& IVariable(<String>) \\
%       &|& INameOf(<ImplementVariable>) \\
%       &|& IValueOf(<ImplementVariable>) \\
%       &|& IFlagOf(<ImplementVariable>) \\
%       &|& SystemIdentifierOf(<ImplementVariable>) \\
%       &|& PublicIdentifierOf(<ImplementVariable>) \\
% \end{eqnarray*}
\end{document}