\documentclass[uplatex,a4j]{jsreport}
\usepackage{thesis}

\begin{document}
\chapter{自然言語処理}
\section{自然言語処理の対象}
 ~\cite{html5specification}
HTML5の字句解析仕様には80個の状態がある。
今回の字句解析仕様の自然言語処理する対象は80個のうち77個とした。
80個の状態のうち、
自然言語処理の対象とした77個の状態は、同じような構造で書かれていた。
しかし残りの、
Markup declaration open state
Named character reference state
Numeric character reference end state
3つの状態はそれぞれ特殊な構造で書かれていたので、これらも一括りにして自然言語処理を適用させるのは、複雑になると判断し、自然言語処理の対象から除外した。

尚、テストする際は残りの3つは手動で実装することにした。

\section{使用ライブラリ}
StanfordCoreNLP
% ライブラリでどういう情報を得られるか
形態素解析(品詞タグ付け、単語の原型の取得),構文解析,意味解析などが出来る.\\
以下の例で詳しく述べる.
\subsection{自然言語処理の例}
"Mika likes her dog's name."をStanfordCoreNLPで自然言語処理をさせる.
\subsubsection{トークン分割、品詞タグ付け、原型}
Mika/NNP  likes/VBZ her/PRP\$ dog/NN 's/POS name/NN ./.
\subsubsection{固有表現抽出}
数字や時間、アドレス、人間、地名といった固有な表現を抽出することが出来る.\\
Mika : PERSON
\subsubsection{構文木解析}
\Tree [.S [.NP [.NNP Mika ] ]
           [.VP
              [.VBZ likes ]
              [.NP [.NP [.PRP\$ her ][.NN dog ][.POS 's ] ]
                    [.NN name ] ]
           ]
           [.. . ]
      ]
\subsubsection{係り受け解析}
係り受け解析とは、単語間の関係を解析するものである.\\
\subsubsection{参照関係の解析}
参照関係の解析とは,文章内で複数個同じものを指し示す単語がある時、それを抽出するものである.itやheなどの指示語の指し示すものを見つける時になどに使用される.\\
Mika, her

\section{対象の前処理}
% htmlパーサーによって自然言語処理する部分を切り分ける
% StateStructureの説明
\subsection{Scala構造体}
HTML5字句解析仕様書は構造的に書かれている.
name,prev,trans
\subsection{文字列の置き換え}
自然言語処理したい文章をそのまま処理すると,トークンの分割や品詞解析が適切な形で解釈されない.\\
よって自然言語処理する際に,前処理として以下の文字列の置き換えをすることによって適切に文章が解釈されるようにした.
\subsubsection*{状態名の置き換え}
attribute value (double-quoted) state =>  
Attribute_value_double_quoted_state 
1つのトークンとして認識させるため、
空白、”-”を”_”にする.”(“,”)”を除く.先頭を大文字にする

\subsubsection*{Unicodeの置き換え}
“U+xxxx” => “U_xxxx”
 (“+”があると2つのトークンに分断されてしまうため)
\subsubsection*{動詞の置き換え}
StanfordCoreNLPは命令文の解釈が苦手である.
品詞解析の時点で動詞と認識されるべき単語が名詞扱いされることがあった.
例えば,"Reconsume"は"re"と"consume"の複合語であり,一般的な辞書にも載ってないので動詞として解釈されないことがあった.
よってこのような単語の前に"you"という単語を付け加え,"you Reconsume $\cdots$"とすることによって,"Reconsume"を動詞として解釈させるようにした.
“<特定の動詞>” => “you <特定の動詞>”
Multiply => multiply

\section{Lemma tree}

\section{Tag型からCommand型への変換}
\subsection*{Switch to the $\cdots$ state}
\Tree [.VP [.VB switch ]
           [.PP
              [.IN to ]
              [.NP $\langle$state$\rangle$ ]
           ]
      ]

$\rightarrow$Switch($\langle$state$\rangle$)
\subsection*{Reconsume in the $\cdots$ state}
\Tree [.VP [.VB reconusme ]
           [.PP
              [.IN in ]
              [.NP $\langle$state$\rangle$ ]
           ]
      ]

$\rightarrow$Reconsume($\langle$state$\rangle$)
\section{If文の処理}

\section{NPノードからCommandValue型への変換}
NPノードをCommandValue型に変換する際,単純に文字列に特定の単語が含まれているかどうかを調べるというやり方で実装した.

\section{参照関係}
% 



\end{document}