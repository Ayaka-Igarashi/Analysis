\documentclass[uplatex,a4j]{jsreport}
\usepackage{thesis}

\begin{document}
\chapter{自然言語処理}
HTML5の字句解析仕様に対して自然言語処理を適用する. 
% 使用ライブラリ
そのための自然言語処理のライブラリとして, スタンフォード大学によって提供されている Stanford CoreNLP~\cite{manning-EtAl:2014:P14-5}を使用し, 処理を行った.

\label{自然言語処理}
\section{自然言語処理の対象}
HTML5の字句解析仕様にはオートマトンとしての状態が80存在する. 
そして, 80個の状態のうち, 77の状態は同じような構造で命令が記述してある. 
しかし, 残りの3状態(
Markup declaration open state, 
Named character reference state, 
Numeric character reference end state
)
はそれぞれ特殊な構造で書かれていたり, 表を参照する必要が出てきたりするので, これらも一括りにして自然言語処理を適用させるのは複雑になると思われた. \\
よって本論文では自然言語処理の適用の対象を80の状態うち, 上記の3状態を除く77の状態に絞ることにした.
尚, テストする際は残りの3状態は手動で実装することにした.

また, HTML5仕様書内のNoteやExample等の補足説明は無視する.

\section{対象の前処理}
HTML5字句解析仕様書はHTMLによって構造的に書かれているので, 仕様書の文章をそのままの状態で丸ごと自然言語処理させると上手くいかない.
よって, 自然言語処理の適用の前段階の処理として, 以下を行うことにした.

字句解析仕様書のHTMLのソースコードを仕様書解析の入力とし, 
前処理の第1段階として, そのソースコードをHTMLパーサーに通し, 仕様書の書かれている構造を認識し, Scalaで定義した構造体に置き換え, 自然言語処理を適用したい部分の文章を抜き出す. 
そして次の段階として, その文章に対して自然言語処理が適切に行われるように, 特定の文字列の置き換えを行う. 

% htmlパーサーによって自然言語処理する部分を切り分ける
\subsection{仕様書の構造の認識, 処理}
仕様書のHTMLソースコードをみると, 1状態あたり, 図\ref{HTMLソースコード}のように書かれている. 
\begin{lstlisting}[basicstyle=\ttfamily\footnotesize, frame=single, caption=HTMLソースコード,label=HTMLソースコード][htbp]
   <h5> <dfn> 状態名 <\dfn> <\h5>
   <p> 文字マッチングする前の処理 <\p>
   <dl>
      <dt> 文字1 <\dt>
      <dd> 文字1にマッチした時の処理 <\dd>
      ...
      <dt> 文字n <\dt>
      <dd> 文字nにマッチした時の処理 <\dd>
   <\dl>
\end{lstlisting}
これを HTMLパーサーのライブラリ jsoup ~\cite{jsoup}を使って, 
「状態名」, 「文字マッチングする前の処理」, 「文字iとその文字にマッチした時の処理のリスト」をもつ, 図\ref{StateStructure}の構造体 State に置き換える. \\
%コードを書く
\begin{lstlisting}[basicstyle=\ttfamily\footnotesize, frame=single, caption=Stateの構造体,label=StateStructure][htbp]
   case class State(name: String, prevProcess: String, trans: List[Trans])
   case class Trans(character: String, process: String)
\end{lstlisting}
% の構造体にHTMLパーサーを使い, 
具体的な処理として, 
``\texttt{h5}'', ``\texttt{dfn}''タグで囲まれている文を``状態名''として取り出し, 
``\texttt{h5}''タグ直後の``\texttt{p}''タグ内の文を``文字マッチングする前の処理''として取り出す.
文字のマッチングの処理は, ``\texttt{dl}''タグ内に書かれており, 
その中の ``\texttt{dt}''タグから``マッチングの文字i'', その直後にある``\texttt{dd}''タグから``文字iにマッチした時の処理''を取り出す.\\
% case class Trans(文字, 文字にマッチした時の処理)

自然言語処理は, 「文字マッチングする前の処理」と「文字iにマッチした時の処理」の部分に対して行う. 


% \subsubsection{StateStructure}
% 状態名は``\texttt{h5}''タグでくくられており,
% 文字マッチング前の処理は``\texttt{h5}''タグ直後の``\texttt{p}''タグ内に記述してある. 
%仕様書に自然言語処理を適用していく.\\
% 1状態あたり, 文字マッチング前の処理, 
% 状態名は``\texttt{h5}''タグ内にある.\\
% 文字マッチング前の処理は``\texttt{h5}''タグ直後の``\texttt{p}''タグ内に記述してある.\\
% 文字マッチングの処理は``\texttt{dt}'', ``\texttt{dd}''タグ.
% StateStructureの説明

% HTML5字句解析仕様のオートマトンとしての状態を1つの単位として, Scalaで定義した, 構造体Stateに変換する.\\
% StateStructure構造体は, 状態名name, 最初の処理prev, 文字マッチングの処理transを持つ.
% 仕様書のHTMLソースファイルの情報をこの構造にする.

\subsection{文字列の置き換え}
自然言語処理したい文章をそのまま処理すると, 様々な原因によりトークンの分割や品詞解析が適切な形で解釈されない
ので, 前処理として以下の文字列の置き換えをすることによって適切に文章が解釈されるようにした.
\subsubsection*{状態名の置き換え}
``script data escape start state''に遷移するという命令文 
``Switch to the script data escape start state.'' は構文木解析において\\
\Tree [.S [.S [.VP [.VB switch ]
              \qroof{to the script data escape}.PP
         ]]
         \qroof{start state}.VP
      ]\\
\vspace{0.5\baselineskip}\\
と ``script data escape'' と ``start state'' が本来同じまとまりの中にいるべき単語がそれぞれ別のまとまりにいると解釈される.\\
また, 状態名の中にカッコや ``-'' がある場合に関しても同じようなことが起こる.\\
よって状態名を1つのトークンとして扱われるようにし, 適切に命令の文が解釈されるようにするため, 仕様書内に出てくる状態名に関して以下の処理をし, 置き換えを行う.
\begin{itemize}
   \item 空白 及び ``\texttt{-}'' を ``\texttt{_}'' に置き換える.
   \item ``('' , ``)'' を除く.
   \item 先頭を大文字にする.
\end{itemize}
先頭を大文字に置き換えるのは, 仕様書内の状態名の先頭が大文字と小文字の場合が混ざっており, 統一させるため.\\
例: 
``attribute value (double-quoted) state'' $\Rightarrow$
``Attribute\_value\_double\_quoted\_state'' 

\subsubsection*{Unicodeの置き換え}
仕様書内では``U+xxxx''というユニコードが多用されている.
これに対して自然言語処理を行うと, トークン分割において``U'', ``+xxxx''と2つのトークンに分割されてしまう.
よってユニコード内の``+''を``\_''に置き換えることによって1つのトークンとして認識させるようにした.\\
例:\\
``U+00AB'' $\Rightarrow$ ``U_00AB''

\subsubsection*{動詞の置き換え}
自然言語処理の出力を確認すると, 品詞解析の時点で動詞と認識されるべき単語が名詞扱いされることがあった.\\
例えば,``Reconsume''は``re''と``consume''の複合語であり, 一般的な辞書にも載ってないので動詞として解釈されないことがあった.
よってこのような単語の前に``you''という単語を付け加え,``you Reconsume $\cdots$''とすることによって,``Reconsume''を動詞として解釈させるようにした.\\
``Reconsume''の他に, 動詞が``Emit'', ``Flush'', ``Append'', ``Multiply''も同じようなことが起こる.
また, Stanford CoreNLPは命令文の解釈が普通の文より上手くいかないことがある.
よって命令文の解釈が上手くいかない文の動詞(``Switch'', ``Append'', ``Multiply'')の前に``you''という仮の主語を付け加え, 命令文にならないようにする.
%“<動詞>” $\Rightarrow$ “you <特定の動詞>”
\begin{itemize}
   \item 文字列 ``Switch'', ``Reconsume'', ``Emit'', ``Flush'', ``Append'', ``Add'', ``Multiply'' の前に``you'' を加える.
\end{itemize}
例: 
``Reconsume in the data state.'' $\Rightarrow$ ``you Reconsume in the data state.''
% Multiply $\Rightarrow$ multiply
\subsubsection*{その他の置き換え}
\begin{itemize}
   \item ``-''で繋がれている単語は1つのトークンとして認識されないため, ``-'' を ``_''に置き換えた.
   \item 句読点をまたいでいる場合,参照関係の解析が上手くいかないことがあった. 参照関係が多く出てくるSet文に関して, 
   ``(,$|$.) set'' $\Rightarrow$ `` and set'' と置き換えをした.
   \item ``!''が文末記号と認識されるため, ``!'' は ``EXC'' に置き換える.
\end{itemize}

% 形態素、構文木の情報、参照関係の情報からTagに変換する過程を書く
\section{命令の抽出元(Tag型)への変換}
命令の抽出をするため, その元となる, 多分木の木構造のデータ型であるTag型をプログラミング言語Scalaで定義し, 
Stanford CoreNLPを用いての自然言語処理から得られる情報のうち,
単語の原型の解析, 構文木解析, 参照関係の解析の情報をTag型への変換に使用した.
%  仕様書の文章が決まった形で書かれていることが多かったため,構文木の情報から命令の抽出が出来ると考え,係り受け解析の情報は使用していない.
% 多分木の木構造のデータ型であるTag型をプログラミング言語Scalaで定義し, それらの情報をTag型に変換した.\\
%  Tagの情報を使用し,命令の抽出を行った.
\subsection{Tag型}
図\ref{Tag}にTag型の定義を載せる.\\
Tag型は, Node型とLeaf型の2種類を持っている.
Node型は構文木の句を表すもので, 句の種類を表すNodeTypeと, そのノードの子であるTag型のリストを持つ.
Leaf型は構文木の末端である単語を表すもので, 品詞名を表すLeafTypeと, 単語の情報を格納するToken型を持つ.
Token型は順に 単語, 単語の原型, 参照関係の番号の情報を持つ.\\
%実際の定義
\begin{lstlisting}[basicstyle=\ttfamily\footnotesize, frame=single, caption=Tagの定義,label=Tag][htbp]
   trait Tag
   case class Node(node: NodeType, list: List[Tag]) extends Tag
   case class Leaf(leaf: LeafType, token: Token) extends Tag
   case class Token(word: String, lemma: String, coref: Int) extends Tag
   trait NodeType
   case object S extends NodeType
   case object NP extends NodeType
   case object VP extends NodeType
   ...
   trait LeafType
   case object NN extends LeafType
   case object NNP extends LeafType
   case object VB extends LeafType
   ...
\end{lstlisting}

\subsubsection{例}
Tag型の値
\begin{lstlisting}[basicstyle=\ttfamily\footnotesize, frame=single][htbp]
   Node(NP, List(Leaf(DT, Token(some,some,-1)), Leaf(NNS, Token(cookies,cookie,-1))))
\end{lstlisting}
を木構造で表記すると, \\
\Tree [.NP [.DT Token(some,some,-1) ]
            [.NNS Token(cookies,cookie,-1) ]
      ]\\
となる.
\subsection{Tag型への変換}
単語の原型の解析, 構文木解析, 参照関係の解析の情報をTag型への変換の方法を書く.
% 変換の対象として,``Create a token. Emit the token.''を例にとる.
%2つのセンテンスに分けられる.\\
\subsubsection{構文木の処理}
\label{構文木の処理}
基本的には自然言語処理の構文木解析で出力された木の形をそのままの状態で同じ木構造であるTag型に変換するが,
例外的に以下の処理を加える.\\%書き方直す
\begin{screen}
1.-NP-PRP-``you''となっている部分を取り除く.\\
2.PRNノード,``(''と``)''の間にあるノードを取り除く.\\
3.ドット(.)を取り除く.\\
4.動詞を表す品詞は複数(VB,VBZ,VBP...)あるが, それらは``VB''に統一する.
\end{screen}
1つ目は, 自然言語の前処理として適切な解釈がなされるように加えた ``you''を取り除くためである.
2つ目の処理は, カッコの中身に書いてある文章は補足説明が多く, 命令の抽出に必要ないと判断したためである.
3つ目は, 既に自然言語処理の段階で文章の分割がなされており不要であるから, Tag構造を簡潔なものにするため取り除く.
4つ目は, 命令の抽出において, 単語が動詞かどうかを判断できれば十分であるので ``VB''に統一することにした.\\

%Leafの説明
Leaf型が持つTokenに関しては, 単語の原型の部分にレンマ化の処理で得た原型の情報を代入し, 参照番号の部分で下記の参照関係の処理を行い, 値を代入する.
% \subsubsection{レンマ化された単語の処理}
% 単語の原型の情報は, 構文木をTag型に変換する際に, その単語のToken型が持つ原型に代入する. %言い方わからん 
\subsubsection{参照関係の処理}
参照関係の出力として, CorefEntity : 1 $\Rightarrow$ [a new tag token, the token]
という風に, 参照関係がある単語と, 参照の番号(この場合では1)が得られる. \\
構文木をTag型に変換する際に, 参照関係を持っている単語のToken型が持つ参照番号をその番号とし, 
参照関係を持たない単語に関しては参照番号を-1とする.\\
%% 変換後のTag構造
\section{Tagへの変換例}
\label{tagEx}
文章 ``Create a comment token. Emit the token.'' からTag型に変換する例を示す.\\
まず, 文字の置き換えの前処理を行うと, ``Create a token. you Emit the token.''となる. \\
次に, これに対して自然言語処理を行うと, \\
単語の原型は, \\
Create/create $\hspace{10pt}$ a/a $\hspace{10pt}$ comment/comment $\hspace{10pt}$ token/token $\hspace{10pt}$ ./.\\
you/you $\hspace{10pt}$ Emit/emit $\hspace{10pt}$ the/the $\hspace{10pt}$ token/token $\hspace{10pt}$ ./.\\
構文木解析は, \\
\Tree [.ROOT [.S [.VP [.VB Create ]
           [.NP
              [.DT a ]
              [.NN comment ]
              [.NN token ]
           ]
      ]
      [.. . ] ] ]
\Tree [.ROOT [.S 
      [.NP [.PRP you ] ]
      [.VP [.VB Emit ]
            [.NP
               [.DT the ]
               [.NN token ]
            ]
      ]
      [.. . ] ] ]\\
\vspace{0.5\baselineskip}\\
参照関係の解析では, 
``a comment token''と ``the token''が同じものであると出力される.\\

そして, 構文木の処理と参照関係の処理を行い, これをTag型への変換をすると, \\
\begin{figure}[h]
   \centering
   \includegraphics[keepaspectratio, scale=0.5]
        {figure/tagTreeRoot.jpg}
   \caption{Tag型}
   \label{tagTree1}
\end{figure}
% 構文木の処理と参照関係の処理を行った結果.\\
% \Tree [.ROOT [.S [.VP [.VB Token(Create,create,-1) ]
%            [.NP
%               [.DT Token(a,a,1) ]
%               [.NN Token(comment,comment,1) ]
%               [.NN Token(token,token,1) ]
%            ]
%       ] ] ]\\
% \Tree [.ROOT [.S [.VP [.VB Token(Emit,emit,-1) ]
%       [.NP
%          [.DT Token(the,the,1) ]
%          [.NN Token(token,token,1) ]
%       ]
%       ] ] ]\\
% \vspace{0.5\baselineskip}\\
図\ref{tagTree1}のようになる.
\end{document}