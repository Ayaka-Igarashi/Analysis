\documentclass[uplatex,a4j]{jsreport}
\usepackage{thesis}

\begin{document}
\chapter{命令の抽出}
\label{命令抽出}
\ref{自然言語処理}章において自然言語処理し, その出力を利用して得たTag型の情報を用いて, \ref{形式}章で定式化した形での命令の抽出を行う.
%% 自然言語処理で取り出した情報(tag構造)を使ってどう命令の抽出を行ったか.
\section{Tag型からCommandへの変換}
Tag型の形に関してのパターンマッチングをして, Command型のリストの値へ変換する.
Tag型のNodeのNodeTypeがS, VP, NP(文, 動詞句, 名詞句)である場合に分け, 変換を行った. 
木の形のマッチに関しては, 葉の値は単語の原型の情報のみ表記, 使用する.\\

この変換を関数として表記する.\\
NodeTypeがSであるNodeから, Command型のリストへ変換する関数を$\mathcal{T_S}$, 
NodeTypeがVPであるNodeから, Command型のリストへ変換する関数を$\mathcal{T}_{VP}$, 
NodeTypeがNPであるNodeから, CommandValue型へ変換する関数を$\mathcal{T}_{NP_{C}}$, 
ImplementVariable型へ変換する関数を$\mathcal{T}_{NP_{I}}$, と書く.

\subsection{文(Sノード)の変換 $\mathcal{T_S}$}
$\mathcal{T_S}$は以下のパターンマッチを行い, マッチしたものに応じたCommand型のリストの値を返す.

\subsubsection{子ノードの先頭がSノード}
Sの子ノードのリストの先頭がSノードの場合, つまりTag型が
\Tree [.S  [.S s1 ]
           [.rst ]
      ]\\
という形(rst はTag型のリスト) \\ $\Rightarrow$ 
$\mathcal{T_S}$(Node(S, s1)) ++ $\mathcal{T_S}$(Node(S, rst)) を返す.\\

\subsubsection{子ノードの先頭がand, then, カンマ}
Sの子ノードのリストの先頭が葉``and'', ``then'', カンマの場合, つまりTag型が\\
\Tree [.S  [.CC and ]
           [.rst ]
      ]
\Tree [.S  [.Comma , ]
            [.rst ]
      ]
\Tree [.S  [.ADVP [.RB then ] ]
           [.rst ]
      ]\\
という形(rst はTag型のリスト) \\ $\Rightarrow$ 
$\mathcal{T}_S$(Node(S, rst)) を返す. 
(``and'', ``then'', カンマは無視し, 子ノードの残りのノードに関して$\mathcal{T}_S$を適用する.)

\subsubsection{子ノードが空リスト}
Sの子ノードの空リストの場合, 空リスト を返す.\\

\subsubsection{子ノードの先頭がVPノード}
Sの子ノードのリストの先頭がVPノードの場合, つまりTag型が
\Tree [.S  [.VP vp1 ]
           [.rst ]
      ]\\
% \Tree [.S [.Node(VP,vp) ] ]\\
%\Tree [.S \qroof{vp}.VP ]\\
という形 (rst はTag型のリスト) \\ $\Rightarrow$ 
% Node Sの子ノードがNode(VP,vp)の時\\
$\mathcal{T}_{VP}$(Node(VP,vp1)) ++ $\mathcal{T_S}$(Node(S,rst)) を返す.\\
(Node(VP, vp)に関して, 動詞句の変換を行う.)\\

%if文書く===
\subsubsection{If文のマッチ}
Sの子ノードのリストの先頭が条件文, つまりTag型が, 
\Tree [.S  [.SBAR [.IN if ] 
                  [.S bool ] ]
           [.rst ]
      ]\\
の形 (rst はTag型のリスト) \\ $\Rightarrow$ 
$[\ $IF_($\mathcal{T_B}$(bool))$\ ]$ ++ $\mathcal{T_S}$(Node(S,rst))\\
を返す.\\
%Otherwise文====
\subsubsection{子ノードの先頭がOtherwise}
Sの子ノードのリストの先頭が``Otherwise''のとき, つまりTag 型が, 
\Tree [.S  [.ADVP [.RB otherwise ] ]
           [.rst ]
      ]\\
の形 (rst はTag型のリスト) \\ $\Rightarrow$ 
$[\ $Otherwise_()$\ ]$ ++ $\mathcal{T_S}$(Node(S,rst)) を返す.\\
%error
\subsubsection{Error文のマッチ}
Tag型が, 
\Tree [.S  [.NP [.DT this ] ]
            [.VP [.VB be ] 
                  \qroof{$\cdots$ error}.NP
            ]
      ]\\
の形 \\ $\Rightarrow$ 
$[\ $Error($\cdots$ error)$\ ]$ を返す.\\

\subsection{動詞句(VPノード)の変換 $\mathcal{T}_{VP}$}
NodeTypeがVPであるNode(動詞句)から, Command型のリストへ変換する関数を$\mathcal{T}_{VP}$と置く.\\
$\mathcal{T}_{VP}$は以下のパターンマッチを行い, マッチしたものに応じたCommand型のListを返す.\\

\subsubsection*{子ノードの先頭がVPノード}
VPの子ノードのリストの先頭がVPノードの場合, つまりTag型が
\Tree [.VP  [.VP vp1 ]
           [.rst ]
      ]\\
という木構造の形の時, 
$\mathcal{T}_{VP}$(Node(VP,vp1))  ++  $\mathcal{T}_{VP}$(Node(VP,rst))  を返す.\\

\subsubsection*{子ノードの先頭がand, カンマ(VPを繋ぐ単語)}
VPの子ノードのリストの先頭が葉``and'', カンマの場合, つまりTag型が\\
\Tree [.VP  [.CC and ]
           [.rst ]
      ]
\Tree [.VP  [.Comma , ]
            [.rst ]
      ]\\
という木構造の形の時, 
$\mathcal{T}_{VP}$(Node(VP, rst)) を返す.\\
\subsubsection*{子ノードが空リスト}
VPの子ノードの空リストの場合, 空のコマンド型のリストを返す.\\

\subsubsection*{Switch文のマッチ}
(元の文 : Switch to the $\cdots$ state)\\
Tagの形が,\\
\Tree [.VP [.VB switch ]
           [.PP
              [.IN to ]
              [.NP np1 ]
           ]
      ]\\
の時, 
$[\ $Switch($\mathcal{T}_{NP_{C}}$(np1)) $\ ]$ を返す.
\subsubsection*{Reconsume文のマッチ}
Reconsume in the $\cdots$ state\\
\Tree [.VP [.VB reconusme ]
           [.PP
              [.IN in ]
              [.NP np1 ]
           ]
      ]\\
の時, 
$[\ $Reconsume($\mathcal{T}_{NP_{C}}$(np1)) $\ ]$ を返す.
% $\rightarrow$Reconsume($\langle$state$\rangle$)

\subsubsection*{Set文のマッチ}
% 変数に値を代入する.
\Tree [.VP [.VB set ]
           [.NP np1 ]
           [.NP
              [.IN to ]
              [.NP np2 ]
           ]
      ]\\
の場合, 
$[\ $Set($\mathcal{T}_{NP_{I}}$(Node(NP, np1)), $\mathcal{T}_{NP_{C}}$(Node(NP, np2))) $\ ]$ を返す. \\
% 変数の状態を変える.
\Tree [.VP [.VB set ]
            [.NP np ]
            [.PP
                [.IN to ]
                [.PP pp ]
            ]
        ]\\
の場合, \\ 
ppが``on''だったら, 
$[\ $Set($\mathcal{T}_{NP}$(Node(NP, np)), On) $\ ]$ を返す. \\
% ``off''だったら, 
% Set($\mathcal{T}_{NP}$(Node(NP, np)), CBool(false)) を返す. \\

\paragraph{個別に対応したSet文のマッチ}
% onと書かれていないflag
``Set the self-closing flag of the current tag token.'' のような状態の切り替え先を示す``to on''が省略されている場合がある.
このような文の構文木は\\
\Tree [.VP [.VB set ]
            [.NP np ]
        ]\\
となる. よってこの形にマッチした場合は, 
$[\ $Set($\mathcal{T}_{NP}$(Node(NP, np)), On) $\ ]$ を返す.\\

% Setのとこ手を加えたやつ===
``Set that attribute's name to the current input character, and its value to the empty string.'' 
みたいな文は木構造が上記のものと違うものになってきてしまうので, これを個別に対応した. \\
木構造が, 
\Tree [.VP [.VB set ]
            [.NP [.NP [.NP np1 ]
                        [.To to ]
                        [.NP np2 ] ]
                  [.Comma , ]
                  [.CC and ]
                  [.NP [.NP np3 ]
                        [.PP [.IN to ]
                              [.NP np4 ] ] ] ]
        ]\\
のとき, \\
$[\ $Set($\mathcal{T}_{NP}$(Node(NP, np1)), $\mathcal{T}_{NP}$(Node(NP, np2))), Set($\mathcal{T}_{NP}$(Node(NP, np3)), $\mathcal{T}_{NP}$(Node(NP, np4)))$\ ]$ を返す. \\
%====

\subsubsection*{AppendTo文のマッチ}
木構造が, 
\Tree [.VP [.VB append ]
            [.NP nplist1 ]
            [.PP [.IN to ]
                  [.NP nplist2 ] ]
 ]
あるいは
\Tree [.VP [.VB append ]
            [.NP [.NP nplist1 ]
                  [.PP [.IN to ]
                        [.NP nplist2 ] ] ]
 ]\\
の場合, 
$[\ $AppendTo($\mathcal{T}_{NP}$(Node(NP, nplist1)), $\mathcal{T}_{NP}$(Node(NP, nplist2))) $\ ]$を返す. \\
\subsubsection*{Emit文のマッチ}
木構造が, 
\Tree [.VP [.VB emit ]
           [.NP nplist ]
      ]\\
の場合, 
% for all np $\leftarrow$ $\mathcal{D}$(Node(NP, nplist)) \\
% \quad $[\ $Emit($\mathcal{T}_{NP}$(np)) $\ ]$ を返す. \\
$\mathcal{D}$(Node(NP, nplist)).map (np $\Rightarrow$ Emit($\mathcal{T}_{NP}$(np))) を返す. \\
木構造が, 
\Tree [.VP [.VB emit ]
      [.NP nplist ]
      \qroof{as a character token}.PP
 ]\\
の場合, 
$[\ $Emit(CharacterToken($\mathcal{T}_{NP}$(Node(NP, nplist)))) $\ ]$ を返す.
\subsubsection*{Consume文のマッチ}
木構造が, 
\Tree [.VP [.VB consume ]
           [.NP nplist ]
      ]\\
の場合, 
$[\ $Consume($\mathcal{T}_{NP}$(Node(NP, nplist)))  $\ ]$を返す.
\subsubsection*{Create文のマッチ}
木構造が, 
\Tree [.VP [.VB create ]
           [.NP nplist ]
      ]\\
の場合 \\ 
nplistの単語に番号nの参照関係が存在 
$\Rightarrow$ 
$[\ $Create(``x$_n$'', $\mathcal{T}_{NP}$(Node(NP, nplist))) $\ ]$を返す. \\
そうでない 
$\Rightarrow$ 
$[\ $Create(``'', $\mathcal{T}_{NP}$(Node(NP, nplist))) $\ ]$を返す.
\subsubsection*{AddTo文のマッチ}
木構造が, 
\Tree [.VP [.VB add ]
            [.NP nplist1 ]
            [.PP [.IN to ]
                  [.NP nplist2 ] ]
 ]
あるいは
\Tree [.VP [.VB add ]
            [.NP [.NP nplist1 ]
                  [.PP [.IN to ]
                        [.NP nplist2 ] ] ]
 ]\\
の場合, 
$[\ $AddTo($\mathcal{T}_{NP}$(Node(NP, nplist1)), $\mathcal{T}_{NP}$(Node(NP, nplist2))) $\ ]$ を返す. \\
\subsubsection*{MultiplyBy文のマッチ}
木構造が, 
\Tree [.VP [.VB multiply ]
            [.NP nplist1 ]
            [.PP [.IN by ]
                  [.NP nplist2 ] ]
 ]\\
 の場合, 
 $[\ $MultiplyBy($\mathcal{T}_{NP}$(Node(NP, nplist1)), $\mathcal{T}_{NP}$(Node(NP, nplist2))) $\ ]$ を返す. \\
\subsubsection*{Ignore, FlushCodePoint文のマッチ}
木構造が, 
\Tree [.VP [.VB v ]
           [.NP $\cdots$ ]
      ]\\
の場合 \\ 
vが "ingore"の時, 
$[\ $Ignore() $\ ]$ を返す. \\
vが "flush"の時, 
$[\ $FlushCodePoint() $\ ]$ を返す.
\subsubsection*{StartAttribute文のマッチ}
木構造が, 
% \Tree [.VP [.VB start ]
%             \qroof{(a new attribute)}.NP
%             \qroof{(in the current tag token)}.PP
% ]\\
\Tree [.VP [.VB start ]
            [.NP $\cdots$ ]
            [.PP $\cdots$ ]
]\\
の場合, 
$[\ $StartAttribute() $\ ]$を返す.
\subsubsection*{TreatAsAnythingElse文のマッチ}
木構造が, 
\Tree [.VP [.VB treat ]
            [.NP $\cdots$ ]
            [.(PP) $\cdots$ ]
            [.ADVP $\cdots$ ]
 ]\\
の場合, 
$[\ $TreatAsAnythingElse() $\ ]$ を返す. \\

\subsection{名詞句(NPノード)の分割 $\mathcal{D}$}
``A and B''や``A, B''など1つの名詞句の中に複数の名詞が含まれている場合もある.
それらを分解し, 名詞のリストとして出力する.

例えば, 
``the current tag token and an end-of-file token''\\
\Tree [.NP 
        [.NP [.DT the ]
              [.JJ current ]
              [.NN tag ]
              [.NN token ] ]
        [.CC and ]
        [.NP [.DT a ]
            [.NN end-of-file ]
            [.NN token ] ]
      ]\\
これを, \\
\Tree [.NP [.DT the ]
              [.JJ current ]
              [.NN tag ]
              [.NN token ] ]
\Tree [.NP [.DT a ]
            [.NN end-of-file ]
            [.NN token ] ]\\
という風にする.
% NPの分解とか(and)
% 名詞句には複数の名詞をandやカンマで区切っているものがある.\\
% それをNPノードのリストに分解した.
% \Tree [.NP 
%         [.NP np1 ]
%         [.CC and ]
%         [.rst ]
%       ]

% \paragraph{特殊な例}
% ``Set that attribute's name and value''のとこ手を加えた. \\
% that attribute'sはnameとvalueどちらにも係っているので, 
% この名詞句は``that attribute's name and that attribute's value''という解釈が正しい.
% しかし, 自然言語処理ではthat attribute'sはnameにしか係っていない風に解釈される.
% \Tree [.NP 
%             \qroof{that attribute 's}.NP
%             [.NN name ]
%             [.CC and ]
%             [.NN value ]
%       ]
    
\subsection{名詞句(NPノード)の変換 $\mathcal{T}_{NP}$}
\subsubsection*{CommandValue型, ImplementVariable型への変換}
HTML5の仕様書には様々な変数の言葉や値の言葉が出てくる. 
NPノード内の自然言語で記述している言葉から, CommandValue型やImplementVariable型というプログラムが認識できる形にする際, 
単純に文字列に変数名や値などの特定の単語が含まれているかどうかを調べるというやり方をとった. 
% NPノードをCommandValue型に変換する際, 

例えば, 変換対象のNPノード内に"return state"という言葉が含まれていた場合は, ReturnStateを返す. 
%参照関係の使用の部分を書く
特定の単語にマッチせず, かつ参照番号が-1でない数字nのとき(その単語にラベルnの参照関係があるとき)は 変数"x\_n"という意味を持つ, Variable("x\_n") を返す.
      name, valueが含まれている場合はNameOf(Variable("x\_n")), ValueOf(Variable("x\_n"))

\section{条件文の処理}
仕様書内の条件分岐の文(If文)に関して, Command型のリストに変換する際, 条件分の範囲を判断するときに手を加えた.\\
% If the temporary buffer is the string ``script'', then switch to the script data escaped state. Otherwise, switch to the script data double escaped state. Emit the current input character as a character token.
例えば, 仕様書内の\\
``If $\cdots$, then $\cdots$. Otherwise, switch to the script data escaped state. Emit the current input character as a character token.''
という文の, ``Emit the current input character as a character token.'' の部分がOtherwiseの中の文として処理したいのか曖昧である. 
このような文は人でも判断しずらいものだから, 機械による処理でも正しく解釈するのは難しい.
よって条件分岐文に関しては, 手動でどこまでがotherwiseにの処理に入る文なのかを判断するようにした. 

具体的な処理としては, 
Tag型からの変換によって出力されたCommandのリストに以下のような形があったら, 
\begin{screen}
      If_(b)\\
      command$_1$\\
      $\cdots$\\
      command$_m$\\
      Otherwise_()\\
      command$_{m+1}$\\
      $\cdots$\\
      command$_n$
\end{screen}
これを考えられるIf文に組み立て, そこから正しいほうを人の手で選ぶ. \\

例えば, 
% \begin{screen}
%       If_(b)\\
%       command$_1$\\
%       Otherwise_()\\
%       command$_2$\\
%       command$_3$
% \end{screen}
条件文の処理をする前のCommand列\\
$[\ $ If_(b), command$_1$, Otherwise_(), command$_2$, command$_3 \ ]$
は, \\
$[\ $ If(b, $[\ $command$_1 \ ]$, $[\ $command$_2$, command$_3 \ ]$ )$\ ]$\\
$[\ $ If(b, $[\ $command$_1 \ ]$, $[\ $command$_2 \ ]$), command$_3 \ ]$ \\
と2つ出力され, 正しい解釈の方を手動で選び, 取り出す命令を決定する.
% どの部分の文章までがotherwiseで処理したい文なのかが曖昧であった.

% 段落を認識させる?

% \section{NPノードからCommandValue型への変換}
% NPノードをCommandValue型に変換する際,単純に文字列に特定の単語が含まれているかどうかを調べるというやり方で実装した.

\section{命令への変換の例}
例\ref{tagEx}の ``Create a comment token. Emit the token.''から命令型Commandのリストを抽出する.\\
``Create a comment token.''から得られた木をT$_1$, ``Emit the token.''から得た木をT$_2$と置く. (ROOTタグは省略)
また, T$_1$, T$_2$の部分木を以下の図\ref{tagTree}のように置く. (点線部分) \\
\begin{figure}[h]
      \centering
      \includegraphics[keepaspectratio, scale=0.45]
           {figure/tagTree.jpg}
      \caption{例\ref{tagEx}で変換されたTag型}
      \label{tagTree}
\end{figure}

% T$_2$は, 子ノードの先頭がVPの場合にマッチするので, \\
% $\mathcal{T}_{S}$(T$_2$) = $\mathcal{T}_{VP}$(T$_5$) ++ $\mathcal{T}_{S}$(Node(S, Nil))\\
% まず, $\mathcal{T}_{VP}$(T$_5$) は, 
% 部分木 T$_5$ がEmit文にマッチするので, 
% $\mathcal{T}_{VP}$(T$_5$) = Emit($\mathcal{T}_{NP}$(T$_6$)) となる. \\
% さらに, $\mathcal{T}_{NP}$(T$_6$) は
% 部分木 T$_6$ がもつ文字列the tokenが番号が1の参照関係を持っているので, 
% $\mathcal{T}_{NP}$(T$_6$) = Variable("x1") となる. \\
% よって, $\mathcal{T}_{S}$(T$_2$) = Emit(Variable("x1"))が得られる.
T$_1$, T$_2$ をそれぞれ$\mathcal{T}_{S}$に適用すると, \\
% \begin{alignat*}{3}
%       &\mathcal{T}_{S}({\rm T}_1) &\quad & = \quad \mathcal{T}_{VP}({\rm T}_3) ++ \mathcal{T}_{S}({\rm Node(S, Nil)})&\quad&  \\
%       & & & = \quad \mathcal{T}_{VP}({\rm T}_3) ++ {\rm Nil} & & \\
%       & & & = \quad {\rm Create}(\mathcal{T}_{NP}({\rm T}_4)) :: {\rm Nil} & & //aaa \\
%       & & & = \quad {\rm Create}({\rm ``x\_1", NewCommentToken}) :: {\rm Nil} & & //aaa\\
%  \end{alignat*}
 \begin{alignat*}{3}
      &\mathcal{T}_{S}({\rm T}_1) &\quad & = \quad \mathcal{T}_{VP}({\rm T}_3) ++\ \mathcal{T}_{S}({\rm Node(S, Nil)})&\quad&  \\
      & & & = \quad \mathcal{T}_{VP}({\rm T}_3) ++\  [\ ] & & \\
      & & & = \quad [\ {\rm Create}(\mathcal{T}_{NP}({\rm T}_4))\ ] & & //\ {\rm T}_3 {\rm はCreate文にマッチする }  \\
      & & & = \quad [\ {\rm Create}({\rm ``x\_1", NewCommentToken})\ ] & & //\  {\rm T}_4{\rm がもつ文字列に``comment\ token"が含まれている.} \\
      & & &                                                              & & \quad\  {\rm その文字列が, 番号が1の参照関係を持っている.}\\
 \end{alignat*}

\begin{alignat*}{3}
      &\mathcal{T}_{S}({\rm T}_2) &\quad & = \quad \mathcal{T}_{VP}({\rm T}_5) ++\ \mathcal{T}_{S}({\rm Node(S, Nil)})&\quad&  \\
      & & & = \quad \mathcal{T}_{VP}({\rm T}_5) ++ \  [\ ] & & \\
      & & & = \quad [\ {\rm Emit}(\mathcal{T}_{NP}({\rm T}_6))\ ] & & //\ {\rm T}_5 {\rm はEmit文にマッチする } \\
      & & & = \quad [\ {\rm Emit}({\rm Variable(``x\_1")})\ ] & & //\ {\rm T}_6{\rm がもつ文字列``the \ token"は番号が1の参照関係を持っている}\\
 \end{alignat*}
となり, 
Command型のリスト\\
$[\ $Create(Variable(``x_1''), NewCommentToken) , Emit(Variable(``x_1''))$\ ]$ % リストの表記
が得られる.

% \subsubsection{gomi}
% ROOTタグ直下のSタグに$\mathcal{T}_{S}$を適用させると, 

% Create(Variable("x1"), NewCommentToken) :: Emit(Variable("x1")) :: Nil
% $\mathcal{T}_{S}$(Node(S, Nil))はNilなので
%   \begin{figure}[h]
%       \centering
%       \includegraphics[keepaspectratio, scale=0.5]
%            {figure/tagTree2.jpg}
%       \label{流れ}
%   \end{figure}
% T$_1 = $ \Tree [.ROOT [.S [.VP [.VB Token(Create,create,-1) ]
%            [.NP
%               [.DT Token(a,a,1) ]
%               [.NN Token(comment,comment,1) ]
%               [.NN Token(token,token,1) ]
%            ]
%       ] ] ]\\
% T$_2 = $  \Tree [.ROOT [.S [.VP [.VB Token(Emit,emit,-1) ]
%                         [.NP
%                         [.DT Token(the,the,1) ]
%                         [.NN Token(token,token,1) ]
%                         ]
%       ] ] ]

% \Tree [.VP [.VB Token(Emit,emit,-1) ]
%             [.NP
%             [.DT Token(the,the,1) ]
%             [.NN Token(token,token,1) ]
%       ] ]\\

% \Tree [.NP
%             [.DT Token(the,the,1) ]
%             [.NN Token(token,token,1) ]
%       ]\\

\section{1状態の形式的な定義}
\label{StateDef}
1状態あたりの処理の内容を記述するクラスを以下で定義する. 
\begin{lstlisting}[basicstyle=\ttfamily\footnotesize, frame=single][htbp]
  case class StateDef(
    stateName:  String,                     // 状態名
    prevProcess :  List[Command],           // 文字マッチ前の処理
    trans :  List[(String, List[Command])]  // 文字マッチ (文字, その文字の処理)のリスト
  )
\end{lstlisting}
\section{例2}
% 1状態からdefinitionの形にする例
\begin{figure}[h]
      \centering
      \includegraphics[keepaspectratio, scale=0.6]
           {figure/RCDATALT.png}
      \caption{RCDATA less-than sign state}
      \label{RCDATA}
\end{figure}
図\ref{RCDATA}の状態に\ref{自然言語処理}, \ref{命令抽出}章の処理をし, \ref{StateDef}節の形式的な定義に変換すると, 
% pStateの形にすると, %形式のとこで説明する?
% pState(stateName: String, prevProcess: List[Command], trans: List[(String, List[Command])])
% stateName = RCDATA_less_than_sign_state
% prevProcess = \\
% $[\ $Consume(NextInputCharacter)$\ ]$
% trans = \\
% $[$\\
% (U+002F SOLIDUS (/), \\
% \quad $[\ $Set(ITemporaryBuffer, CString(``'')), Switch(RCDATA_end_tag_open_state)$\ ]$), \\
% (Anything else, \\
% \quad $[\ $Emit(CString(``$<$'')), Recomsume(RCDATA_state)$\ ]$)\\
% $]$\\

\begin{alignat*}{3}
      &{\rm stateName }&\quad &{\rm = \quad RCDATA\_less\_than\_sign\_state } {\rm \quad\quad\quad//\  状態名 }&\quad & \\
      &{\rm prevProcess }& &{\rm = \quad}[\  {\rm Consume(NextInputCharacter) }\ ] {\rm \ \ \quad\quad//\  文字マッチ前の処理 }& & \\
      &{\rm //\  文字マッチ }&\quad & &\quad & \\
      &{\rm trans }& &{\rm = \quad}[\ & & \\
      & & &{\rm \quad\quad\quad  (\ } \texttt{U+002F} {\rm \ SOLIDUS\ (/)\ ,} & & \\
      & & &{\rm \quad\quad\quad\quad} [\ {\rm Set(ITemporaryBuffer,\ CString()),\ Switch(RCDATA\_end\_tag\_open\_state)} \ ]\ )\ ,& & \\
      & & &{\rm \quad\quad\quad} (\ {\rm Anything else\ ,} & & \\
      & & &{\rm \quad\quad\quad\quad} [\ {\rm Emit(CString(<)),\ Recomsume(RCDATA\_state)} \ ]\ )& & \\
      & & &{\rm \quad\quad ]}& & \\
\end{alignat*}
となる.
\end{document}