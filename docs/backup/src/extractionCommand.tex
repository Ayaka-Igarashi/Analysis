\documentclass[uplatex,a4j]{jsreport}
\usepackage{thesis}

\begin{document}
\chapter{命令の抽出}
\label{命令抽出}
\ref{自然言語処理}章において自然言語処理し, その出力を利用して得たTag型の情報を用いて, \ref{形式}章で定式化した形での命令の抽出を行う.
%% 自然言語処理で取り出した情報(tag構造)を使ってどう命令の抽出を行ったか.
\section{Tag型からCommandへの変換}
Tag型の値の木構造の形に関してのパターンマッチをし, 形式的な命令へ変換する.
% Tag型のNodeのNodeTypeがS, VP, NP(文, 動詞句, 名詞句)である場合に分け, 変換を行った. 
Tag型からCommandの形式への変換を関数として表記する.
\begin{alignat*}{3}
      &\mathcal{T}_S&\ :\quad &{\rm Tag}\ \rightarrow\ {\rm List\ [Command ] }&\quad\quad &// \ {\rm 文(NodeTypeがSであるNode)から, }\\
      & & & & &\hspace{13pt} {\rm Command型のリストへ変換する関数}\\
      &\mathcal{T}_B&\ :\quad &{\rm Tag}\ \rightarrow\ {\rm Bool }& &// \ {\rm 条件文(NodeTypeがSであるNode)から,}\\
      & & & & &\hspace{13pt} {\rm  Bool型へ変換する関数}\\
      &\mathcal{T}_{VP}&\ :\quad &{\rm Tag}\ \rightarrow\ {\rm List\ [ Command ] }& &// \ {\rm 動詞句(NodeTypeがVPであるNode)から,}\\
      & & & & &\hspace{13pt} {\rm Command型のリストへ変換する関数}\\
      &\mathcal{D}&\ :\quad &{\rm Tag}\ \rightarrow\ {\rm List\ [ Tag ]} & &// \ {\rm 名詞句(NodeTypeがNPであるNode)から,}\\
      & & & & &\hspace{13pt} {\rm  1単位ごとの名詞句に分割する関数}\\
      &\mathcal{T}_{NP_{C}}&\ :\quad &{\rm Tag}\ \rightarrow\ {\rm CommandValue }& &// \ {\rm 名詞句(NodeTypeがNPであるNode)から,}\\
      & & & & &\hspace{13pt} {\rm  CommandValue型へ変換する関数}\\
      &\mathcal{T}_{NP_{I}}&\ :\quad &{\rm Tag}\ \rightarrow\ {\rm InplementVariable} & &// \ {\rm 名詞句(NodeTypeがNPであるNode)から, }\\
      & & & & &\hspace{13pt} {\rm ImplementVariable型へ変換する関数}\\
\end{alignat*}

これらを順に説明していく.
木の形のマッチに関しては, 葉の値は単語の原型の情報のみ表記, 使用する.\\
\subsection{文(Sノード)の変換 $\mathcal{T}_S$}
$\mathcal{T}_S$はTag型の値 T を受け取り, Tの形にマッチしたものに応じたCommand型のリストの値を返す. \\
% 以下のパターンマッチを行い, マッチしたものに応じたCommand型のリストの値を返す.
\subsubsection{文, 句の分解}
複数の文, 句から構成されている時, 
Stagの子ノードを先頭から順に処理していく. \\
% 子ノードの先頭から順に処理していく.\\
根の子ノードの先頭のノード形によって場合分け\\
Tag型の値 T の形が, 
\begin{figure}[H]
      \centering
      \includegraphics[keepaspectratio, scale=0.55]
           {figure/SnodeList.jpg}
\end{figure}
(i) の場合,  $\mathcal{T}_S$(T$^\prime$) ++ $\mathcal{T}_S$(T$_{rst}$) を返す.\\
(ii) の場合,  $\mathcal{T}_{VP}$(T$^\prime$) ++ $\mathcal{T}_S$(T$_{rst}$) を返す. 
(木 T$^\prime$に関して, 動詞句の変換を行う.)\\
(iii) の場合,  $\mathcal{T}_S$(T$_{rst}$) を返す. 
(``and''などの文を繋ぐ単語は無視する.)\\
(iv) の場合,  [\ ] を返す. (空のCommnad型のリスト)\\
\subsubsection{条件分岐文の場合}
Tag型の値 T の形が, 
\begin{figure}[H]
      \centering
      \includegraphics[keepaspectratio, scale=0.55]
           {figure/ifOther.jpg}
\end{figure}
(i) の場合,  $[\ $If_($\mathcal{T}_B$(T$^\prime$))$\ ]$ ++ $\mathcal{T}_S$(T$_{rst}$) を返す.\\
(ii) の場合,  $[\ $Otherwise_()$\ ]$ ++ $\mathcal{T}_S$(Node(S,rst)) を返す.\\

% \subsubsection{子ノードの先頭がSノード}
% Sの子ノードのリストの先頭がSノードの場合, つまりTag型が
% \begin{figure}[H]
%       \includegraphics[keepaspectratio, scale=0.55]
%            {figure/snode.jpg}
% \end{figure}
% % \Tree [.S  [.S s1 ]
% %            [.rst ]
% %       ]\\
% という形 
% % (rst はTag型のリスト) \\ 
% $\Rightarrow$ 
% % $\mathcal{T}_S$(Node(S, s1)) ++ $\mathcal{T}_S$(Node(S, rst)) を返す.\\
% $\mathcal{T}_S$(T$^\prime$) ++ $\mathcal{T}_S$(T$_{rst}$) を返す.

% \subsubsection{子ノードの先頭がand, then, カンマ}
% Sの子ノードのリストの先頭が葉``and'', ``then'', カンマの場合, つまりTag型が\\
% \begin{figure}[H]
%       \includegraphics[keepaspectratio, scale=0.55]
%            {figure/sAnd.jpg}
% \end{figure}
% % \Tree [.S  [.CC and ]
% %            [.rst ]
% %       ]
% % \Tree [.S  [.Comma , ]
% %             [.rst ]
% %       ]
% % \Tree [.S  [.ADVP [.RB then ] ]
% %            [.rst ]
% %       ]\\
% という形 
% % (rst はTag型のリスト) \\ 
% $\Rightarrow$ 
% $\mathcal{T}_S$(T$_{rst}$) を返す. 
% % $\mathcal{T}_S$(Node(S, rst)) を返す. 
% (``and'', ``then'', カンマは無視し, 子ノードの残りのノードに関して$\mathcal{T}_S$を適用する.)

% \subsubsection{子ノードが空リスト}
% Sの子ノードの空リストの場合, 空リスト を返す.

% \subsubsection{子ノードの先頭がVPノード}
% Sの子ノードのリストの先頭がVPノードの場合, つまりTag型が
% \begin{figure}[H]
%       \includegraphics[keepaspectratio, scale=0.55]
%            {figure/svp.jpg}
% \end{figure}
% % \Tree [.S  [.VP vp1 ]
% %            [.rst ]
% %       ]\\
% % \Tree [.S [.Node(VP,vp) ] ]\\
% %\Tree [.S \qroof{vp}.VP ]\\
% という形 
% % (rst はTag型のリスト) \\ 
% $\Rightarrow$ 
% $\mathcal{T}_{VP}$(T$^\prime$) ++ $\mathcal{T}_S$(T$_{rst}$) を返す. 
% % $\mathcal{T}_{VP}$(Node(VP,vp1)) ++ $\mathcal{T}_S$(Node(S,rst)) を返す.\\
% (木 T$^\prime$に関して, 動詞句の変換を行う.)

%if文書く===
% \subsubsection{If文のマッチ}
% Sの子ノードのリストの先頭が条件文, つまりTag型が,
% \begin{figure}[H]
%       \includegraphics[keepaspectratio, scale=0.55]
%            {figure/sif.jpg}
% \end{figure} 
% % \Tree [.S  [.SBAR [.IN if ] 
% %                   [.S bool ] ]
% %            [.rst ]
% %       ]\\
% の形 
% % (rst はTag型のリスト) \\ 
% $\Rightarrow$ 
% $[\ $If_($\mathcal{T}_B$(T$^\prime$))$\ ]$ ++ $\mathcal{T}_S$(T$_{rst}$) 
% を返す.
%Otherwise文====
% \subsubsection{子ノードの先頭がOtherwise}
% Sの子ノードのリストの先頭が``Otherwise''のとき, つまりTag 型が, 
% \Tree [.S  [.ADVP [.RB otherwise ] ]
%            [.rst ]
%       ]\\
% の形 (rst はTag型のリスト) \\ $\Rightarrow$ 
% $[\ $Otherwise_()$\ ]$ ++ $\mathcal{T}_S$(Node(S,rst)) を返す.
%error
\subsubsection{Error文のマッチ}
Tag型の値 T の形が, 
\begin{figure}[H]
      \centering   
      \includegraphics[keepaspectratio, scale=0.55]
           {figure/error.jpg}
\end{figure}
% \Tree [.S  [.NP [.DT this ] ]
%             [.VP [.VB be ] 
%                   \qroof{$\cdots$ error}.NP
%             ]
%       ]\\
の時,  
$[\ $Error($\cdots$ error)$\ ]$ を返す.\\
\subsection{条件文の変換 $\mathcal{T}_{B}$}
$\mathcal{T}_B$はTag型の値 T を受け取り, T の形にマッチしたものに応じたBool型の値を返す. \\
\begin{figure}[H]
      \centering   
      \includegraphics[keepaspectratio, scale=0.55]
           {figure/isEqual.jpg}
\end{figure}
部分木T$_1$ 内の限定詞を除いた文字列が ``current end tag token'', 
部分木T$_2$ 内の限定詞を除いた文字列が ``appropriate end tag token'' の時, 
CurrentEndTagIsAppropriate()\\
T$_2$の形が, 
\begin{figure}[H]
      \centering   
      \includegraphics[keepaspectratio, scale=0.55]
           {figure/matchfor.jpg}
\end{figure}
の場合, 
AsciiCaseInsensitiveMatch($\mathcal{T}_{NP_{C}}$(T$_1$), $\mathcal{T}_{NP_{C}}$(T$_2^\prime$))\\
それ以外の時, 
IsEqual($\mathcal{T}_{NP_{C}}$(T$_1$), $\mathcal{T}_{NP_{C}}$(T$_2$))\\

また, 木T内にある文字列が の文字列が, 
``character reference was consumed as part of an attribute'' の時, 
CharacterReferenceConsumedAsAttributeVal() を返す.
\subsection{動詞句(VPノード)の変換 $\mathcal{T}_{VP}$}
$\mathcal{T}_{VP}$はTag型の値 T を受け取り, Tの形にマッチしたものに応じたCommand型のリストの値を返す. 

% NodeTypeがVPであるNode(動詞句)から, Command型のリストへ変換する関数を$\mathcal{T}_{VP}$と置く.\\
% $\mathcal{T}_{VP}$は以下のパターンマッチを行い, マッチしたものに応じたCommand型のListを返す.\\
ここで表記していない, 残りのCommand型の命令に関しても, その命令文の構文木の形を調べて, 変換の規則を作っていく.

\subsubsection{動詞句の分解}
子ノードに動詞句を持つパターン\\
根の子ノードの先頭のノード形によって場合分け\\
Tag型の値 T の形が, 
\begin{figure}[H]
      \centering
      \includegraphics[keepaspectratio, scale=0.55]
           {figure/vplist.jpg}
\end{figure}
(i) の場合, $\mathcal{T}_{VP}$(T$^\prime$) ++ $\mathcal{T}_{VP}$(T$_{rst}$) を返す.\\
(ii) の場合,  $\mathcal{T}_{VP}$(T$_{rst}$) を返す. 
(``and''などの文を繋ぐ単語は無視する.)\\
(iii) の場合,  [\ ] (空のCommnad型のリスト) を返す.\\

% \subsubsection*{子ノードの先頭がVPノード}
% VPの子ノードのリストの先頭がVPノードの場合, つまりTag型が
% \begin{figure}[H]
%       \includegraphics[keepaspectratio, scale=0.55]
%            {figure/vpvp.jpg}
% \end{figure}
% % \Tree [.VP  [.VP vp1 ]
% %            [.rst ]
% %       ]\\
% という木構造の形の時, 
% $\mathcal{T}_{VP}$(T$^\prime$) ++ $\mathcal{T}_{VP}$(T$_{rst}$)  を返す.\\
% % $\mathcal{T}_{VP}$(Node(VP,vp1))  ++  $\mathcal{T}_{VP}$(Node(VP,rst))  を返す.\\
% \subsubsection*{子ノードの先頭がand, カンマ(VPを繋ぐ単語)}
% VPの子ノードのリストの先頭が葉``and'', カンマの場合, つまりTag型が\\
% \Tree [.VP  [.CC and ]
%            [.rst ]
%       ]
% \Tree [.VP  [.Comma , ]
%             [.rst ]
%       ]\\
% という木構造の形の時, 
% $\mathcal{T}_{VP}$(Node(VP, rst)) を返す.\\
% \subsubsection*{子ノードが空リスト}
% VPの子ノードの空リストの場合, 空のコマンド型のリストを返す.\\
\subsubsection*{Switch文のマッチ}
命令文の構文木解析の形を調べ, それに基づいたTag型の値のパターンマッチを作る.
% (元の文 : Switch to the $\cdots$ state)\\
Tagの形が,
\begin{figure}[H]
      \centering   
      \includegraphics[keepaspectratio, scale=0.55]
           {figure/switch.jpg}
\end{figure}
% \Tree [.VP [.VB switch ]
%            [.PP
%               [.IN to ]
%               [.NP np1 ]
%            ]
%       ]\\
の時, 
$[\ $Switch($\mathcal{T}_{NP_{C}}$(T$^\prime$)) $\ ]$ を返す.
% \subsubsection*{Reconsume文のマッチ}
% Reconsume in the $\cdots$ state\\
% \Tree [.VP [.VB reconsume ]
%            [.PP
%               [.IN in ]
%               [.NP np1 ]
%            ]
%       ]\\
% の時, 
% $[\ $Reconsume($\mathcal{T}_{NP_{C}}$(np1)) $\ ]$ を返す.
\subsubsection*{Emit文のマッチ}
Tag型の値Tの木構造が, 
\begin{figure}[H]
      \centering
      \includegraphics[keepaspectratio, scale=0.55]
           {figure/emit.jpg}
\end{figure}
% \Tree [.VP [.VB emit ]
%            [.NP nplist ]
%       ]\\
(i)の場合, 
% for all np $\leftarrow$ $\mathcal{D}$(Node(NP, nplist)) \\
% \quad $[\ $Emit($\mathcal{T}_{NP}$(np)) $\ ]$ を返す. \\
$\mathcal{D}$(T$^\prime$) = [T$^\prime_1$, $\ldots$, T$^\prime_n$] の時, 
$[\ $Emit($\mathcal{T}_{NP}$(T$^\prime_1$)), $\ldots$,Emit($\mathcal{T}_{NP}$(T$^\prime_n$))  $\ ]$ を返す. \\
% $\mathcal{D}$(Node(NP, nplist)).map (np $\Rightarrow$ Emit($\mathcal{T}_{NP}$(np))) を返す. \\
% 木構造が, 
% \begin{figure}[H]
%       \includegraphics[keepaspectratio, scale=0.55]
%            {figure/emit2.jpg}
% \end{figure}
% \Tree [.VP [.VB emit ]
%       [.NP nplist ]
%       \qroof{as a character token}.PP
%  ]\\
(ii)の場合, 
$\mathcal{D}$(T$^\prime$) = [T$^\prime_1$, $\ldots$, T$^\prime_n$] の時, \\
$[\ $Emit(CharacterToken($\mathcal{T}_{NP}$(T$^\prime_1$))), $\ldots$,Emit(CharacterToken($\mathcal{T}_{NP}$(T$^\prime_n$)))  $\ ]$ を返す. \\
% $[\ $Emit(CharacterToken($\mathcal{T}_{NP}$(Node(NP, nplist)))) $\ ]$ を返す.
\subsubsection*{Set文のマッチ}
木構造が, 
\begin{figure}[H]
      \centering
      \includegraphics[keepaspectratio, scale=0.55]
           {figure/set.jpg}
\end{figure}
% 変数に値を代入する.
% \Tree [.VP [.VB set ]
%            [.NP np1 ]
%            [.NP
%               [.IN to ]
%               [.NP np2 ]
%            ]
%       ]\\
(i)の場合, 変数に値を代入するという意味のSet文である. \\
$\mathcal{D}$(T$_1$) = [T$_1^1$, $\ldots$, T$_1^n$] の時, 
$[\ $Set($\mathcal{T}_{NP_{I}}$(T$_1^1$), $\mathcal{T}_{NP_{C}}$(T$_2$)), $\ldots$,Set($\mathcal{T}_{NP_{I}}$(T$_1^n$), $\mathcal{T}_{NP_{C}}$(T$_2$)) $\ ]$ を返す. \\
% $[\ $Set($\mathcal{T}_{NP_{I}}$(T$_1$), $\mathcal{T}_{NP_{C}}$(T$_2$)) $\ ]$ を返す. \\
% 変数の状態を変える.
% \Tree [.VP [.VB set ]
%             [.NP np ]
%             [.PP
%                 [.IN to ]
%                 [.PP pp ]
%             ]
%         ]\\
(ii)の場合, 変数の状態を変えるという意味のSet文である. \\
部分木T$_2$内にある文字列が``on''だったら, 
$\mathcal{D}$(T$_1$) = [T$_1^1$, $\ldots$, T$_1^n$] の時, 
$[\ $Set($\mathcal{T}_{NP_{I}}$(T$_1^1$), On), $\ldots$,Set($\mathcal{T}_{NP_{I}}$(T$_1^n$), On) $\ ]$ を返す. \\
% $[\ $Set($\mathcal{T}_{NP_{I}}$(T$_1$), On) $\ ]$ を返す. \\
% ``off''だったら, 
% Set($\mathcal{T}_{NP}$(Node(NP, np)), CBool(false)) を返す. \\

\subsubsection*{個別に対応したSet文のマッチ}
% Setのとこ手を加えたやつ===
``Set that attribute's name to the current input character, and its value to the empty string.'' 
のような, 1文の中に2つの命令が含まれている文章も仕様書内に出てくるが, このような文の構文木は, 
% みたいな文は木構造が上記のものと違うものになってきてしまうので, これを個別に対応した. \\
% 木構造が, 
\begin{figure}[H]
      \centering   
      \includegraphics[keepaspectratio, scale=0.55]
           {figure/set1234.jpg}
\end{figure}
のように, 特殊な形になっている. よってこのような木構造になっているもののCommand型の変換を個別に対応し, 
% \Tree [.VP [.VB set ]
%             [.NP [.NP [.NP np1 ]
%                         [.To to ]
%                         [.NP np2 ] ]
%                   [.Comma , ]
%                   [.CC and ]
%                   [.NP [.NP np3 ]
%                         [.PP [.IN to ]
%                               [.NP np4 ] ] ] ]
%         ]\\
% のとき, \\
% $[\ $Set($\mathcal{T}_{NP}$(Node(NP, np1)), $\mathcal{T}_{NP}$(Node(NP, np2))), Set($\mathcal{T}_{NP}$(Node(NP, np3)), $\mathcal{T}_{NP}$(Node(NP, np4)))$\ ]$ を返す. \\
$[\ $Set($\mathcal{T}_{NP}$(T$_1$), $\mathcal{T}_{NP}$(T$_2$)), Set($\mathcal{T}_{NP}$(T$_3$), $\mathcal{T}_{NP}$(T$_4$))$\ ]$ を返すようにした. 

% onと書かれていないflag
また, 
``Set the self-closing flag of the current tag token.'' のような状態の切り替え先を示す``to on''が省略されている文章も仕様書内に存在する. 
このような文に対応するため, 
Tag型の値Tが, 
\begin{figure}[H]
      \centering
      \includegraphics[keepaspectratio, scale=0.55]
           {figure/setflag.jpg}
\end{figure}
% \Tree [.VP [.VB set ]
%             [.NP np ]
%         ]\\
にマッチした場合は, 
$[\ $Set($\mathcal{T}_{NP}$(Node(NP, np)), On) $\ ]$ を返すようにした. 

% \subsubsection*{AppendTo文のマッチ}
% 木構造が, 
% \Tree [.VP [.VB append ]
%             [.NP nplist1 ]
%             [.PP [.IN to ]
%                   [.NP nplist2 ] ]
%  ]
% あるいは
% \Tree [.VP [.VB append ]
%             [.NP [.NP nplist1 ]
%                   [.PP [.IN to ]
%                         [.NP nplist2 ] ] ]
%  ]\\
% の場合, 
% $[\ $AppendTo($\mathcal{T}_{NP}$(Node(NP, nplist1)), $\mathcal{T}_{NP}$(Node(NP, nplist2))) $\ ]$を返す. 

% \subsubsection*{Consume文のマッチ}
% 木構造が, 
% \Tree [.VP [.VB consume ]
%            [.NP nplist ]
%       ]\\
% の場合, 
% $[\ $Consume($\mathcal{T}_{NP}$(Node(NP, nplist)))  $\ ]$を返す.
\subsubsection*{Create文のマッチ}
木構造が, 
\begin{figure}[H]
      \centering
      \includegraphics[keepaspectratio, scale=0.55]
           {figure/create.jpg}
\end{figure}
% \Tree [.VP [.VB create ]
%            [.NP nplist ]
%       ]\\
の場合 \\ 
T$^\prime$の単語に番号nの参照関係が存在 
$\Rightarrow$ 
$[\ $Create(``x$_n$'', $\mathcal{T}_{NP}$(T$^\prime$)) $\ ]$を返す. \\
そうでない 
$\Rightarrow$ 
$[\ $Create(``'', $\mathcal{T}_{NP}$(T$^\prime$)) $\ ]$を返す.
% \subsubsection*{AddTo文のマッチ}
% 木構造が, 
% \Tree [.VP [.VB add ]
%             [.NP nplist1 ]
%             [.PP [.IN to ]
%                   [.NP nplist2 ] ]
%  ]
% あるいは
% \Tree [.VP [.VB add ]
%             [.NP [.NP nplist1 ]
%                   [.PP [.IN to ]
%                         [.NP nplist2 ] ] ]
%  ]\\
% の場合, 
% $[\ $AddTo($\mathcal{T}_{NP}$(Node(NP, nplist1)), $\mathcal{T}_{NP}$(Node(NP, nplist2))) $\ ]$ を返す. \\
% \subsubsection*{MultiplyBy文のマッチ}
% 木構造が, 
% \Tree [.VP [.VB multiply ]
%             [.NP nplist1 ]
%             [.PP [.IN by ]
%                   [.NP nplist2 ] ]
%  ]\\
%  の場合, 
%  $[\ $MultiplyBy($\mathcal{T}_{NP}$(Node(NP, nplist1)), $\mathcal{T}_{NP}$(Node(NP, nplist2))) $\ ]$ を返す. \\
% \subsubsection*{Ignore, FlushCodePoint文のマッチ}
% 木構造が, 
% \Tree [.VP [.VB v ]
%            [.NP $\cdots$ ]
%       ]\\
% の場合 \\ 
% vが "ingore"の時, 
% $[\ $Ignore() $\ ]$ を返す. \\
% vが "flush"の時, 
% $[\ $FlushCodePoint() $\ ]$ を返す.
% \subsubsection*{StartAttribute文のマッチ}
% 木構造が, 
% % \Tree [.VP [.VB start ]
% %             \qroof{(a new attribute)}.NP
% %             \qroof{(in the current tag token)}.PP
% % ]\\
% \Tree [.VP [.VB start ]
%             [.NP $\cdots$ ]
%             [.PP $\cdots$ ]
% ]\\
% の場合, 
% $[\ $StartAttribute() $\ ]$を返す.
% \subsubsection*{TreatAsAnythingElse文のマッチ}
% 木構造が, 
% \Tree [.VP [.VB treat ]
%             [.NP $\cdots$ ]
%             [.(PP) $\cdots$ ]
%             [.ADVP $\cdots$ ]
%  ]\\
% の場合, 
% $[\ $TreatAsAnythingElse() $\ ]$ を返す. \\

\subsection{名詞句(NPノード)の分割 $\mathcal{D}$}
$\mathcal{D}$はTag型の値 T を受け取り, それを分割したTag型のリストの値を返す. 

% 分割と変換を一つの関数にまとめたほうがいいかも
% NPノードを受け取って, NPノードのリストを返す関数. \\
名詞句には, 
``A and B''や``A, B''のような, 複数の名詞句の連結の形であることがある. 
%など1つの名詞句の中に複数の名詞が含まれている場合もある.
$\mathcal{D}$はそれらを分解し, Tag型のリストとして出力するものである. 

例えば, 以下のandで繋がれている名詞句を
``the current tag token and an end-of-file token''\\
\Tree [.NP 
        [.NP [.DT the ]
              [.JJ current ]
              [.NN tag ]
              [.NN token ] ]
        [.CC and ]
        [.NP [.DT a ]
            [.NN end-of-file ]
            [.NN token ] ]
      ]\\
以下の2つの名詞句のTag型に分解する. \\
\Tree [.NP [.DT the ]
              [.JJ current ]
              [.NN tag ]
              [.NN token ] ]
\Tree [.NP [.DT a ]
            [.NN end-of-file ]
            [.NN token ] ]\\
% NPの分解とか(and)
% 名詞句には複数の名詞をandやカンマで区切っているものがある.\\
% それをNPノードのリストに分解した.
% \Tree [.NP 
%         [.NP np1 ]
%         [.CC and ]
%         [.rst ]
%       ]

% \paragraph{特殊な例}
% ``Set that attribute's name and value''のとこ手を加えた. \\
% that attribute'sはnameとvalueどちらにも係っているので, 
% この名詞句は``that attribute's name and that attribute's value''という解釈が正しい.
% しかし, 自然言語処理ではthat attribute'sはnameにしか係っていない風に解釈される.
% \Tree [.NP 
%             \qroof{that attribute 's}.NP
%             [.NN name ]
%             [.CC and ]
%             [.NN value ]
%       ]
    
\subsection{名詞句(NPノード)の変換 $\mathcal{T}_{NP_C}$, $\mathcal{T}_{NP_I}$}
$\mathcal{T}_{NP_C}$, $\mathcal{T}_{NP_I}$はTag型の値 T を受け取り, それぞれCommandValue型, ImplementVariable型の値を返す. 
$\mathcal{T}_{NP_C}$と$\mathcal{T}_{NP_I}$は変換後の型が違うのみで, 同じ動作をする. 

Tag型の値Tから, CommandValue型やImplementVariable型へ変換する際, 
T内の文字列に変数名や値などを表す特定の文字列が含まれているかどうかと, 何らかの参照関係が存在するかを調べるというやり方をとった. 
% \subsubsection*{CommandValue型, ImplementVariable型への変換}
% HTML5の仕様書には様々な変数の言葉や値の言葉が出てくる. 
% NPノード内の自然言語で記述している言葉から, CommandValue型やImplementVariable型というプログラムが認識できる形にする際, 
% ノード内に含まれる文字列に, 変数名や値などを表す特定の文字列が含まれているかどうかと, 
% そのノード内の文字列に何らかの参照関係が存在するかを調べるというやり方をとった. 
% NPノードをCommandValue型に変換する際, 

例えば, T内に"return state"という文字列が含まれていた場合は, $\mathcal{T}_{NP_C}$の場合はReturnState, $\mathcal{T}_{NP_I}$の場合はIReturnStateを返す. \\

%参照関係の使用の部分を書く
あるCommandValue型の値cvalに変換され, 参照番号が-1でない数字nのとき(その単語にラベルnの参照関係があるとき)は, 
字句解析器の実行において, CommandValue型の意味を解釈する際, 変数"x\_n"に解釈後のcvalの値を代入するという意味を持つ, 
Substitute("x\_n", cval) を返す. \\

特定の単語にマッチせず, かつ参照番号が-1でない数字nのとき(その単語にラベルnの参照関係があるとき)は 変数"x\_n"という意味を持つ, Variable("x\_n") を返す.
また, name, valueが含まれており, 参照番号が-1でない数字nの場合はNameOf(Variable("x\_n")), ValueOf(Variable("x\_n")) を返す.

\subsubsection{例}
Tag型の値が
\begin{figure}[H]
      \centering
      \includegraphics[keepaspectratio, scale=0.55]
           {figure/itstagname.jpg}
\end{figure}
の場合, 
% \Tree [.NP  [.JJ Token(its) ]
%             [.NN tag ]
%             [.NN name ] ]
この文字列には, nameという単語が含まれているかつ, ``its''に参照関係の番号1がついているので, NameOf(Variable("x\_1")) を返す. 

\section{条件分岐文の処理}
Command型への変換の時に, %If文をまとめてあれしなかったのは, 
If(...,...,...)の様に, If文の内容をまとめて取り出すのではなく, If_(...), ..., Otherwise_(), ...という風に順に一つの要素ずつ取り出したのは, 
If(...,...,...)という形に, 手動で組み立てようと考えからである. 
この方針にした理由は, Otherwiseの処理がどこまでなのかの判別が難しかったからである.
% ``If $\cdots$ then $\cdots$. Otherwise $\cdots$.''という決まった形である
% 仕様書内の条件分岐の文(If文)に関して, Command型のリストに変換する際, 条件分の範囲を判断するときに手を加えた.\\
% If the temporary buffer is the string ``script'', then switch to the script data escaped state. Otherwise, switch to the script data double escaped state. Emit the current input character as a character token.
例えば, 仕様書内の\\
``If $\cdots$, then $\cdots$. Otherwise, switch to the script data escaped state. Emit the current input character as a character token.''
という文の, ``Emit the current input character as a character token.'' の部分がOtherwiseの中の文として処理したいのか曖昧である. 
このような文は人でも判断しずらいものもあった. %だから, 機械による処理でも正しく解釈するのは難しい.
よって条件分岐文に関しては, 手動でどこまでがotherwiseにの処理に入る文なのかを判断するようにした. 

具体的な処理としては, 
Tag型からの変換によって出力されたCommandのリストに \\
$[\ $ If_(b), command$_1$, $\ldots$ , command$_m$, Otherwise_(), command$_{m+1}$, $\ldots$ , command$_n \ ]$\\
% \begin{screen}
%       If_(b)\\
%       command$_1$\\
%       $\cdots$\\
%       command$_m$\\
%       Otherwise_()\\
%       command$_{m+1}$\\
%       $\cdots$\\
%       command$_n$
% \end{screen}
のような形があったら, これを考えられるIf文に組み立て, そこから正しいほうを人の手で選ぶ. 
\subsection*{例}
% \begin{screen}
%       If_(b)\\
%       command$_1$\\
%       Otherwise_()\\
%       command$_2$\\
%       command$_3$
% \end{screen}
条件文の処理をする前のCommand列\\
% $[\ $ If_(b), command$_1$, Otherwise_(), command$_2$, command$_3 \ ]$
$[\ $ If_(b), Switch($\cdots$), Otherwise_(), Switch($\cdots$), Emit($\cdots$)$ \ ]$
は, \\
% \begin{equation}
%       \begin{cases*}
%             [\  {\rm If(b, }[\ {\rm command}_1 \ ],\ [\ {\rm command}_2,\ {\rm command}_3 \ ] )\ ]\\
%             [\  {\rm If(b, }[\ {\rm command}_1 \ ],\ [\ {\rm command}_2 \ ] )\ ,\ \ {\rm command}_3\ ]
%       \end{cases*}
% \end{equation}
% $\Rightarrow\ $ $[\ $ If(b, $[\ $command$_1 \ ]$, $[\ $command$_2$, command$_3 \ ]$ )$\ ]$\\
% $\Rightarrow\ $ $[\ $ If(b, $[\ $command$_1 \ ]$, $[\ $command$_2 \ ]$), command$_3 \ ]$ \\
$\Rightarrow\ $ $[\ $ If(b, $[\ $Switch($\cdots$)$ \ ]$, $[\ $Switch($\cdots$), Emit($\cdots$)$ \ ]$ )$\ ]$\\
$\Rightarrow\ $ $[\ $ If(b, $[\ $Switch($\cdots$)$ \ ]$, $[\ $Switch($\cdots$)$ \ ]$), Emit($\cdots$)$ \ ]$ \\
と2つ出力され, 正しい解釈の方を手動で選び, 取り出す命令を決定する.
% どの部分の文章までがotherwiseで処理したい文なのかが曖昧であった.

% 段落を認識させる?

\section{命令への変換の例}
例\ref{tagEx}の ``Create a comment token. Emit the token.''から命令型Commandのリストを抽出する.\\
``Create a comment token.''から得られた木をT$_1$, ``Emit the token.''から得た木をT$_2$と置く. (ROOTタグは省略)
また, T$_1$, T$_2$の部分木を以下の図\ref{tagTree}のように置く. (点線部分) \\
\begin{figure}[h]
      \centering
      \includegraphics[keepaspectratio, scale=0.45]
           {figure/tagTree.jpg}
      \caption{例\ref{tagEx}で変換されたTag型}
      \label{tagTree}
\end{figure}

% T$_2$は, 子ノードの先頭がVPの場合にマッチするので, \\
% $\mathcal{T}_{S}$(T$_2$) = $\mathcal{T}_{VP}$(T$_5$) ++ $\mathcal{T}_{S}$(Node(S, Nil))\\
% まず, $\mathcal{T}_{VP}$(T$_5$) は, 
% 部分木 T$_5$ がEmit文にマッチするので, 
% $\mathcal{T}_{VP}$(T$_5$) = Emit($\mathcal{T}_{NP}$(T$_6$)) となる. \\
% さらに, $\mathcal{T}_{NP}$(T$_6$) は
% 部分木 T$_6$ がもつ文字列the tokenが番号が1の参照関係を持っているので, 
% $\mathcal{T}_{NP}$(T$_6$) = Variable("x1") となる. \\
% よって, $\mathcal{T}_{S}$(T$_2$) = Emit(Variable("x1"))が得られる.
T$_1$, T$_2$ をそれぞれ$\mathcal{T}_{S}$に適用すると, \\
% \begin{alignat*}{3}
%       &\mathcal{T}_{S}({\rm T}_1) &\quad & = \quad \mathcal{T}_{VP}({\rm T}_3) ++ \mathcal{T}_{S}({\rm Node(S, Nil)})&\quad&  \\
%       & & & = \quad \mathcal{T}_{VP}({\rm T}_3) ++ {\rm Nil} & & \\
%       & & & = \quad {\rm Create}(\mathcal{T}_{NP}({\rm T}_4)) :: {\rm Nil} & & //aaa \\
%       & & & = \quad {\rm Create}({\rm ``x\_1", NewCommentToken}) :: {\rm Nil} & & //aaa\\
%  \end{alignat*}
 \begin{alignat*}{3}
      &\mathcal{T}_{S}({\rm T}_1) &\quad & = \quad \mathcal{T}_{VP}({\rm T}_3) ++\ \mathcal{T}_{S}({\rm Node(S, Nil)})&\quad&  \\
      & & & = \quad \mathcal{T}_{VP}({\rm T}_3) ++\  [\ ] & & \\
      & & & = \quad [\ {\rm Create}(\mathcal{T}_{NP}({\rm T}_4))\ ] & & //\ {\rm T}_3 {\rm はCreate文にマッチする }  \\
      & & & = \quad [\ {\rm Create}({\rm ``x\_1", NewCommentToken})\ ] & & //\  {\rm T}_4{\rm がもつ文字列に``comment\ token"が含まれている.} \\
      & & &                                                              & & \quad\  {\rm その文字列が, 番号が1の参照関係を持っている.}\\
 \end{alignat*}

\begin{alignat*}{3}
      &\mathcal{T}_{S}({\rm T}_2) &\quad & = \quad \mathcal{T}_{VP}({\rm T}_5) ++\ \mathcal{T}_{S}({\rm Node(S, Nil)})&\quad&  \\
      & & & = \quad \mathcal{T}_{VP}({\rm T}_5) ++ \  [\ ] & & \\
      & & & = \quad [\ {\rm Emit}(\mathcal{T}_{NP}({\rm T}_6))\ ] & & //\ {\rm T}_5 {\rm はEmit文にマッチする } \\
      & & & = \quad [\ {\rm Emit}({\rm Variable(``x\_1")})\ ] & & //\ {\rm T}_6{\rm がもつ文字列``the \ token"は番号が1の参照関係を持っている}\\
 \end{alignat*}
となり, 
Command型のリスト\\
$[\ $Create(Variable(``x_1''), NewCommentToken) , Emit(Variable(``x_1''))$\ ]$ % リストの表記
が得られる.

% \subsubsection{gomi}
% ROOTタグ直下のSタグに$\mathcal{T}_{S}$を適用させると, 

% Create(Variable("x1"), NewCommentToken) :: Emit(Variable("x1")) :: Nil
% $\mathcal{T}_{S}$(Node(S, Nil))はNilなので
%   \begin{figure}[h]
%       \centering
%       \includegraphics[keepaspectratio, scale=0.55]
%            {figure/tagTree2.jpg}
%       \label{流れ}
%   \end{figure}
% T$_1 = $ \Tree [.ROOT [.S [.VP [.VB Token(Create,create,-1) ]
%            [.NP
%               [.DT Token(a,a,1) ]
%               [.NN Token(comment,comment,1) ]
%               [.NN Token(token,token,1) ]
%            ]
%       ] ] ]\\
% T$_2 = $  \Tree [.ROOT [.S [.VP [.VB Token(Emit,emit,-1) ]
%                         [.NP
%                         [.DT Token(the,the,1) ]
%                         [.NN Token(token,token,1) ]
%                         ]
%       ] ] ]

% \Tree [.VP [.VB Token(Emit,emit,-1) ]
%             [.NP
%             [.DT Token(the,the,1) ]
%             [.NN Token(token,token,1) ]
%       ] ]\\

% \Tree [.NP
%             [.DT Token(the,the,1) ]
%             [.NN Token(token,token,1) ]
%       ]\\

\section{1状態の形式的な定義}
\label{StateDef}
実装では, 
1状態あたりの形式化した内容を記述するクラスとして\texttt{StateDef}を定義する. 
\texttt{StateDef}は状態名, 文字マッチ前の処理, 文字マッチの処理を有する. 
\begin{lstlisting}[basicstyle=\ttfamily\footnotesize, frame=single][htbp]
  case class StateDef(
    stateName:  String,                     // 状態名
    prevProcess :  List[Command],           // 文字マッチ前の処理
    trans :  List[(String, List[Command])]  // 文字マッチ (文字, その文字の処理)のリスト
  )
\end{lstlisting}
\subsection{例}
% 1状態からdefinitionの形にする例
% \begin{figure}[h]
%       \centering
%       \fbox{\includegraphics[keepaspectratio, scale=0.6]
%       {figure/RCDATALT.png}
%       }
%       \caption{RCDATA less-than sign state}
%       \label{RCDATA}
% \end{figure}
\begin{table}[htb]
      \begin{center}
        \caption{HTML5字句解析仕様の状態 : RCDATA less-than sign state}
          \begin{tabular}{|p{2.5cm}|p{3.5cm}|p{8.5cm}|}\hline
              {\bf 状態名} & \multicolumn{2}{|p{12cm}|}{RCDATA less-than sign state}\\ \hline
              {\bf 文字マッチ前の処理} & \multicolumn{2}{|p{12cm}|}  { Consume the next input character: } \\ \hline
              {\bf 文字マッチ処理} & {\bf 文字} & {\bf 処理} \\ \cline{2-3}
              & \begin{tabular}{l}U+002F \\SOLIDUS (/) \end{tabular}& Set the temporary buffer to the empty string. Switch to the RCDATA end tag open state. \\ \cline{2-3}
              & \begin{tabular}{l}Anything else \end{tabular}& Emit a U+003C LESS-THAN SIGN character token. Reconsume in the RCDATA state. \\ \hline
          \end{tabular}
          \label{state9}
      \end{center}
  \end{table}
表\ref{state9}の状態に\ref{自然言語処理}, \ref{命令抽出}章の処理をし, 1状態の形式的な定義に変換すると, 
図\ref{RCDATA形式化}のようになる. 
% pStateの形にすると, %形式のとこで説明する?
% pState(stateName: String, prevProcess: List[Command], trans: List[(String, List[Command])])
% stateName = RCDATA_less_than_sign_state
% prevProcess = \\
% $[\ $Consume(NextInputCharacter)$\ ]$
% trans = \\
% $[$\\
% (U+002F SOLIDUS (/), \\
% \quad $[\ $Set(ITemporaryBuffer, CString(``'')), Switch(RCDATA_end_tag_open_state)$\ ]$), \\
% (Anything else, \\
% \quad $[\ $Emit(CString(``$<$'')), Recomsume(RCDATA_state)$\ ]$)\\
% $]$\\

% \begin{alignat*}{3}
%       &{\rm stateName }&\quad &{\rm = \quad RCDATA\_less\_than\_sign\_state } {\rm \quad\quad\quad//\  状態名 }&\quad & \\
%       &{\rm prevProcess }& &{\rm = \quad}[\  {\rm Consume(NextInputCharacter) }\ ] {\rm \ \ \quad\quad//\  文字マッチ前の処理 }& & \\
%       &{\rm //\  文字マッチ }&\quad & &\quad & \\
%       &{\rm trans }& &{\rm = \quad}[\ & & \\
%       & & &{\rm \quad\quad\quad  (\ } \texttt{U+002F} {\rm \ SOLIDUS\ (/)\ ,} & & \\
%       & & &{\rm \quad\quad\quad\quad} [\ {\rm Set(ITemporaryBuffer,\ CString()),\ Switch(RCDATA\_end\_tag\_open\_state)} \ ]\ )\ ,& & \\
%       & & &{\rm \quad\quad\quad} (\ {\rm Anything else\ ,} & & \\
%       & & &{\rm \quad\quad\quad\quad} [\ {\rm Emit(CString(<)),\ Recomsume(RCDATA\_state)} \ ]\ )& & \\
%       & & &{\rm \quad\quad ]}& & \\
% \end{alignat*}
\begin{lstlisting}[basicstyle=\ttfamily\footnotesize, frame=single, caption=RCDATA less-than sign stateの形式化, label=RCDATA形式化][htbp]
   // 状態名
   stateName = RCDATA_less_than_sign_state
   // 文字マッチ前の処理
   prevProcess = [ Consume(NextInputCharacter) ]
   // 文字マッチ
   trans = [
      ( U+002F SOLIDUS (/), 
            [ Set(ITemporaryBuffer, CString()), Switch(RCDATA_end_tag_open_state) ] ),
      ( Anything else, 
            [ Emit(CString(<)), Recomsume(RCDATA_state) ] )
   ]
\end{lstlisting}
\end{document}