\documentclass[uplatex,a4j]{jsreport}
\usepackage{thesis}

\begin{document}
\chapter{結論}
命令の種類が限られており,それぞれ決まった書き方をしていることが多かったので,
構文木の情報のみでも命令の抽出がやりやすかった.(機械的な文字のマッチングでも出来そうではあった.)\\
しかし今回はやらなかったが,特に命令の記法が一貫していない場合は係り受け解析を用いたほうが様々な形式の文章に対応できるので良いと思われる.

%%
% ある程度仕様書の記法に一貫性があれば、上手くできる.
% it等の指示語が多く使われていたから、参照関係の解析をするという面では自然言語処理は役に立つ.
% ユニコードの記述など仕様書で固有表現が抽出されなかったので、前処理が必要だった.
\end{document}