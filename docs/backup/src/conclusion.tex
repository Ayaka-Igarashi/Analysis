\documentclass[uplatex,a4j]{jsreport}
\usepackage{thesis}

\begin{document}
\chapter{結論}
% \section{まとめ}
本論文では, HTML5の字句解析仕様に自然言語処理を適用させ, 命令の抽出が行えることを確認した. 
本研究で扱った仕様に関しては, 命令の種類や記述の仕方が限られており, 構文木の情報から命令の抽出が可能だった. 

しかし, 反省点として, 
構文木から命令型への変換において, 構文木の形が標準の形と違ってきてしまう文章に対して, 個別に対応する必要が出てきてしまったり, 
名詞句のTag型の値からCommandValue型への変換が単に文字列のマッチングをするという愚直なやり方になってしまった部分がある. 

%それぞれ決まった書き方をしていることが多かったので, 
%(機械的な文字のマッチングでも出来そうではあった.)\\
その解決策として, 構文木解析の情報のみではなく係り受け解析の情報も利用する方法があると思われる. 
% しかし今回はやらなかったが, 特に命令の記法が一貫していない場合は係り受け解析を用いたほうが様々な形式の文章に対応できるので良いと思われる.

%係り受け解析を用いたほうが便利だと思った話.
例えば, Tag型から命令の型であるCommand型へ変換する際, 
木構造のマッチングにおいて, 
Emit文のマッチの部分で, 
``Emit $\langle$Emitの対象$\rangle$''のような単純な文ではなく, ``Emit the current input character as a character token.'' のように ``as a character token''という補足的な情報が加わると, 構文木の形が変わるので, 複数種類のパターンマッチをする必要が出てきた. 
% といった文がある. この文のように``current input character''に``as a character token''のような補足的な情報が加わると, 複数種類のパターンマッチ文を書く必要が出てきており, 
% 命令抽出の対象の記法が比較的一貫したので煩雑さは抑えられたが, 
その点において, 
係り受け解析から得られる情報だとEmitする対象となる文字を直接知ることが出来るので, 係り受け解析を用いたほうが簡潔に命令を抽出できる場合があると感じた. \\

また, 名詞句のTag型の値からCommandValue型への変換する際に関しても, 
%係り受け解析を使用したほうが楽だと思った.
例えば, ``its tag name''という名詞句を例にとると, \\
この文の構文木解析は, \\
\Tree [.NP [.PRP\$ its ]
            [.NN tag ]
            [.NN name ]
     ]\\
という結果だが, 
係り受け解析だと, \\
\begin{figure}[H]
    \centering
    \includegraphics[keepaspectratio, scale=0.6]
         {figure/tagname.png}
    \label{npEx}
\end{figure}
という結果が得られ, 
``name''が``its''のものであるということがはっきりわかり, 得られる情報が多い.\\
よって, 名詞句を解析する際, 構文木解析と係り受け解析併用したほうがいいと思った. \\
% 正直、NPノードを変換する奴はかなり力技でやってしまった。という反省


%掛っている部分にかんして.
% that attribute'sはnameとvalue


%
%%
% ある程度仕様書の記法に一貫性があれば、上手くできる.
% it等の指示語が多く使われていたから、参照関係の解析をするという面では自然言語処理は役に立つ.
% ユニコードの記述など仕様書で固有表現が抽出されなかったので、前処理が必要だった.
\end{document}