\documentclass[uplatex,a4j]{jsreport}
\usepackage{thesis}

\begin{document}
\chapter{結論}
命令の種類が限られており,それぞれ決まった書き方をしていることが多かったので,
構文木の情報のみでも命令の抽出が可能だった. (機械的な文字のマッチングでも出来そうではあった.)\\
しかし今回はやらなかったが, 特に命令の記法が一貫していない場合は係り受け解析を用いたほうが様々な形式の文章に対応できるので良いと思われる.

% いい感じに書く
%係り受け解析を用いたほうが便利だと思った.
例えば, Tag型から命令の型であるCommand型へ変換する際, 
構文木のマッチングで, 
``Emit the current input character as a character token.''
といった文がある. この文のように``current input character''に``as a character token''のような補足的な情報が加わると, 複数種類のパターンマッチ文を書く必要が出てきており, 
命令抽出の対象の記法が比較的一貫したので煩雑さは抑えられたが, 
係り受け解析を用いたほうが簡潔にできると感じた.
%掛っている部分にかんして.
that attribute'sはnameとvalue

%%
% ある程度仕様書の記法に一貫性があれば、上手くできる.
% it等の指示語が多く使われていたから、参照関係の解析をするという面では自然言語処理は役に立つ.
% ユニコードの記述など仕様書で固有表現が抽出されなかったので、前処理が必要だった.
\end{document}