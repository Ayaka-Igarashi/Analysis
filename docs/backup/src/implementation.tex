\documentclass[uplatex,a4j]{jsreport}
\usepackage{thesis}

\begin{document}
\chapter{実装}
\label{実装}
\ref{命令抽出}章でHTML5字句解析仕様から抽出し, 形式化したもの使ってHTML5の字句解析のインタープリタを作成した.
\section{概要}
% インタープリタの概略図書く
\begin{figure}[h]
    \centering
    \includegraphics[keepaspectratio, scale=0.5]
         {figure/インタープリタ.png}
    \caption{インタープリタ概略}
    \label{interpret}
\end{figure}
形式化した命令をもとに動かす. 
\section{インタープリタの実装の詳細}
\subsection{CommandValue型の解釈}
CommandValue型と環境を受け取り, Value型を返す関数を実装した.
\subsubsection{Value型}
インタープリタで扱う値の型
\begin{lstlisting}[basicstyle=\ttfamily\footnotesize, frame=single, caption=Value型,label=Value][htbp]
    IntVal(int: Int)
    BooleanVal(boolean: Boolean)
    CharVal(c: Char)
    StringVal(string: String)
    EOFVal
    StateVal(statename: String)
    TokenVal(token: Token)
\end{lstlisting}
\subsection{Command型の解釈}
Command型と環境を受け取り, 新しい環境, 排出トークン, エラー内容を返す関数を実装した.

% \section{字句解析器の変数}
% next state

% current state

% return state


% \section{Command型}
% 実装したCommand型のそれぞれの動作を操作的意味論を用いて表す. 
% \subsection*{Switch(state: CommandValue)}
% $\langle$ Switch(state), $env \rangle \rightarrow env[nextState \leftarrow \mathcal{I}_{cval}[\![state]\!] ]$

% \subsection*{Recomsume(state: CommandValue)}
% $\langle$ Recomsume(state), $env \rangle \rightarrow env[nextState \leftarrow \mathcal{I}_{cval}[\![state]\!] , {\rm inputText} \leftarrow {\rm char + inputText} ]$\\
%  if currentInputCharacter = CharVal(char)\\
% $\langle$ Recomsume(state), $env \rangle \rightarrow env[nextState \leftarrow \mathcal{I}_{cval}[\![state]\!] , {\rm inputText} \leftarrow {\rm string + inputText} ]$\\
%  if currentInputCharacter = StringVal(string)\\
% $\langle$ Recomsume(state), $env \rangle \rightarrow env[nextState \leftarrow \mathcal{I}_{cval}[\![state]\!] ]$\\
%  if currentInputCharacter = EOFVal\\

% \subsection*{Consume()}

% \subsection*{Consume()}

% \subsection*{Consume()}

% \subsection*{Consume()}

% \subsection*{Consume()}

% \subsection*{Consume()}

% \subsection*{Consume()}

% \subsection*{Consume()}

% \subsection*{Consume()}

% \subsection*{Consume()}

% \subsection*{If(bool: Bool, t: CommandList, f: CommandList)}
% \begin{prooftree}
%     \AxiomC{$\langle$ clist1, $env\rangle \rightarrow env^\prime $ }
%     \RightLabel{{\scriptsize if $\mathcal{B}[\![b]\!] = true$}}
%     \UnaryInfC{$\langle$if $b$ then clist1 else clist2, $env\rangle \rightarrow env^\prime$}
% \end{prooftree}
% \begin{prooftree}
%     \AxiomC{$\langle$ clist2, $env\rangle \rightarrow env^\prime $ }
%     \RightLabel{{\scriptsize if $\mathcal{B}[\![\mathcal{C}[\![b]\!]]\!] = false$}}
%     \UnaryInfC{$\langle$if $b$ then clist1 else clist2, $env\rangle \rightarrow env^\prime$}
% \end{prooftree}

% \section{Bool型}
% \subsection*{And(a: Bool, b: Bool)}
% \subsection*{CharacterReferenceConsumedAsAttributeVal()}
% \subsection*{CurrentEndTagIsAppropriate()}
% \subsection*{IsEqual(a: CommandValue, b: CommandValue)}

% \section{Token型}
% tagToken(isStart: Boolean, name: String, attributes: List[Attribute])
% DOCTYPEToken( systemIdentifier: String, publicIdentifier: String)
% characterToken()


% \section{CommandValue型}
% CommandValue型からValue型の値を返す関数\\
% $C : {\rm CommandValue} \rightarrow {\rm Value} $\\
% \subsection*{LowerCaseVersion(cVal: CommandValue)}
% c.toLowerCase if $C[\![$cVal$]\!] = $c: Char or String
% \subsection*{NumericVersion(cVal: CommandValue)}
% Integer.parseInt(c.toString, 16)
% \subsection*{NextInputCharacter}
% $
% \begin{cases}
%     {\rm CharVal(c)} & {\rm inputText.headOption} = {\rm Some(c)} \\
%     {\rm EOFVal} & {\rm inputText.headOption} = {\rm None}
% \end{cases}
% $
% \subsection*{CurrentInputCharacter}
% currentInputCharacter
% \subsection*{EndOfFileToken}
% TokenVal(endOfFileToken())

% \section{ImplementVariable型}
% $\mathcal{I}_{ival}$
% \subsection{IReturnState}
% returnState

\end{document}