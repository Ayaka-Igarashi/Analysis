\documentclass[uplatex,a4j]{jsreport}
\usepackage{thesis}

\begin{document}
\chapter{評価}
\section{HTML5テスト}
\subsection*{テスト結果}
contentModelFlags => 24/24
domjs => 42/58 (((okikae? ,if, CDATAの分岐(42)
 entities => 80/80 -> 状態80のテスト
 escapeFlag => 9/9
 namedEntities => 4210/4210 -> 状態73のテスト
 numericEntities => 336/336 -> 状態73のテスト
 pendingSpecChanges => 1/1
test1 => 63/68  !!!(((if,
 test2 => 35/45  !!!(((public,system -> 45/45
 test3 => 1374/1786  !!!(((pub,sys ->1786/1786
 test4 => 81/85  !!!((( public  -> 85/85
 unicodeChars => 323/323
 unicodeCharsProblem => 5/5
xml => 1/4  (((okikae
\section{上手くいかなかった点}
If the six characters starting from the current input character are an ASCII case-insensitive match for the word "PUBLIC", then consume those characters
この文章を自然言語解析させると"those characters"は"the six characters starting from the current input character"を参照するという出力になる.\\
もし,この状態へ遷移した時点での入力文字列が"public $\cdots$"であったら,まず文字'p'を消費し,入力文字列が"ublic $\cdots$"となる.\\
機械的にこの文章を処理しようとすると,現在の入力文字列"ublic $\cdots$"から文字列"public"を消費せよという解釈になるので,上手くいかない.\\
\end{document}