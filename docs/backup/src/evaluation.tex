\documentclass[uplatex,a4j]{jsreport}
\usepackage{thesis}

\begin{document}
\chapter{評価}
\label{評価}
\section{HTML5テスト}
字句解析のインタプリターの正しさを検証するために,html5lib-tests~\cite{html5lib-tests}のtokenizerのテストデータを用い,テストを行った.
\subsection*{テスト結果}
\begin{table}[htb]
    \begin{tabular}{|l|c|l|} \hline
      テストファイル名 & 結果 & テスト内容\\ \hline 
      contentModelFlags.test & 24/24 & あ\\
      domjs.test & 50/58 & あ\\
      entities.test & 80/80 & あ\\
      escapeFlag.test & 9/9 & あ\\
      namedEntities.test & 4210/4210 & あ\\
      numericEntities.test & 336/336 & あ\\
      pendingSpecChanges.test & 1/1 & あ\\
      test1.test & 68/68 & あ\\
      test2.test & 35/45 & あ\\
      test3.test & 1374/1786 & あ\\
      test4.test & 81/85 & あ\\
      unicodeChars.test & 323/323 & あ\\
      unicodeCharsProblem.test & 5/5 & あ\\ \hline 
    \end{tabular}
\end{table}

\section{思ったこと}
命令抽出に関しては, 手作業でやった部分が多いので上手くいったと思う.

\section{問題点}
%上手くいかなかった点
\subsection*{test2.test,test3.test,test4.testが上手くいかなかった原因}%正しく命令が抽出されたように思われても、正しくなかった例
If the six characters starting from the current input character are an ASCII case-insensitive match for the word ``PUBLIC'', then consume those characters
この文章を自然言語解析させると``those characters''は``the six characters starting from the current input character''を参照するという出力になる.\\
もし,この状態へ遷移した時点での入力文字列が``public $\cdots$''であったら,まず文字'p'を消費し,入力文字列が``ublic $\cdots$''となる.\\
機械的にこの文章を処理しようとすると,現在の入力文字列``ublic $\cdots$''から文字列``public''を消費せよという解釈になるので,上手くいかない.\\
この問題を手動で解決させた結果,以下のようなテスト結果の改善が成された.\\
\begin{table}[htb]
    \begin{tabular}{|l|c|} \hline
      テストファイル名 & 結果 \\ \hline 
      test2.test & 45/45 \\
      test3.test & 1786/1786 \\
      test4.test & 85/85 \\ \hline 
    \end{tabular}
\end{table}

\subsection*{domjsが上手くいかなかった原因}
入力文字列を字句解析器に通す前に, 入力文字列に対して文字の置き換えをする必要があった.

\end{document}