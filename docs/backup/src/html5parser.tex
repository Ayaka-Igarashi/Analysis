\documentclass[uplatex,a4j]{jsreport}
\usepackage{thesis}

\begin{document}
\chapter{HTML5字句解析仕様}
\label{字句解析仕様}
まず, 自然言語処理の適用対象である, HTML5の字句解析仕様について概要を述べていく.
\section{概要}
%トークンより状態の説明先がいい?
% \subsection*{状態}
HTML5字句解析仕様はWHATWG communityのwebサイト~\cite{html5specification}から得られ, 
全て自然言語(英語)によって書かれている. 

HTML5の字句解析器は80個の状態のあるオートマトンとして定義されており, 
% それぞれの状態は以下の形式で処理が記述されている. 
ほとんどの状態は, 文字マッチ前の処理と文字マッチングの処理(文字とその文字にマッチした時の処理)の要素から成り立っている. 
表\ref{state72}は字句解析器の状態の1つである, Character reference state の例である.
% \begin{figure}[h]
%     \centering
%     \fbox{\includegraphics[keepaspectratio, scale=0.5]
%             {figure/html5-3.png}
%     }    
%     \caption{HTML5字句解析仕様}
%     \label{html5}
% \end{figure}
\begin{table}[htb]
    \begin{center}
      \caption{HTML5字句解析仕様の状態 : Character reference state}
        \begin{tabular}{|p{2.5cm}|p{3.5cm}|p{8.5cm}|}\hline
            {\bf 状態名} & \multicolumn{2}{|p{12cm}|}{Character reference state}\\ \hline
            {\bf 文字マッチ前の処理} & \multicolumn{2}{|p{12cm}|}  {  Set the temporary buffer to the empty string. Append a U+0026 AMPERSAND (\&) character to the temporary buffer. Consume the next input character: } \\ \hline
            {\bf 文字マッチ処理} & {\bf 文字} & {\bf 処理} \\ \cline{2-3}
            & \begin{tabular}{l}ASCII \\alphanumeric \end{tabular}& Reconsume in the named character reference state. \\ \cline{2-3}
            & \begin{tabular}{l}U+0023 \\NUMBER SIGN (\#) \end{tabular}& Append the current input character to the temporary buffer. Switch to the numeric character reference state. \\ \cline{2-3}
            & \begin{tabular}{l}Anything else \end{tabular}& Flush code points consumed as a character reference. Reconsume in the return state. \\ \hline
        \end{tabular}
        \label{state72}
    \end{center}
\end{table}
% 状態を1つの単位として, 仕様書に自然言語処理を適用していく.
\subsection*{字句解析トークン}
HTML5字句解析器は, HTML5で書かれている文章をその文法の構成単位であるトークンに分解するプログラムである.\\
HTML5字句解析器により出力されるトークンは5つの種類がある.
文書で利用するHTMLやXHTMLのバージョンを表すトークンである, DOCTYPEトークン(DOCTYPE token), 
``\texttt{<!-- -->}''などでコメントアウトした文章を値として持つコメントトークン(comment token), 
`\texttt{<}'と`\texttt{>}'で囲まれている, HTMLのタグを表す開始タグトークン(start tag token)と終了タグトークン(end tag token), 
文字の情報を表す文字トークン(character token), 文章の終了を表すEOFトークン(end-of-file token)が存在する.

EOFトークン以外はそれぞれ値を持っている.\\
DOCTYPEトークンはDOCTYPEの名前,  公開識別子(public identifier)とシステム識別子(system identifier), 強制互換モードのフラグ(force-quirks flag)の要素を持っている.\\
開始,終了タグトークンはタグの名前, 属性の集合(attributes), セルフクロージングタグかどうかのフラグ(self-closing flag)の要素を持っている.\\
文字トークンは文字のデータを持つ. 
コメントトークンはコメント内容の文字列データを持つ.\\
% EOFトークンが排出されたら字句解析器の動作は終了する.\\

具体的なトークンへの分解の例として, HTML5文章
\begin{lstlisting}[basicstyle=\ttfamily\footnotesize, frame=single][htbp]
      <!DOCTYPE html> <!-- ABCDEFG --> <p> hello </p>
\end{lstlisting}
をトークンに分解すると, \\
DOCTYPEトークン(名前: \texttt{html}, 公開識別子: null, システム識別子: null, 強制互換モードのフラグ: off)\\
コメントトークン(\texttt{ABCDEF})\\
開始タグトークン(名前: \texttt{p}, 属性: 無し, セルフクロージングタグのフラグ: off)\\
文字トークン(\texttt{hello})\\
終了タグトークン(名前: \texttt{p}, 属性: 無し, セルフクロージングタグのフラグ: off)\\
となる. \\

字句解析器により出力されたトークンはDOMツリーを構成する次の構文解析のステップに使われる.

\subsection*{字句解析器の環境の変数}
HTML5字句解析器はreturn stateや一時バッファなどの様々な変数を持ち, 様々な命令により, 環境の変数の値を変化させていき, 動作する.
% これだけで大丈夫なのか?

\section{字句解析器の動作例}
% 字句解析器の動作をステップごとに分けて記述していく.
\subsection{例1}
入力``\texttt{<a>bc</a>}''に対して, HTML5字句解析器は以下のように動作を行う.\\
\begin{enumerate}
    \item 初期状態Data stateから, 文字'\texttt{<}'が消費され, Tag open state に遷移する.
    \item 文字'\texttt{a}'を消費し, 名前が空文字である新たな開始タグトークンを作る. Tag name stateに遷移する.
    \item 先ほど消費した文字'\texttt{a}'を再度消費し, 開始タグトークンの名前に'\texttt{a}'を付け足す.
    \item 文字'\texttt{>}'を消費し, 開始タグトークンを排出し, Data stateに遷移する.
    \item 文字'\texttt{b}'を消費し, 文字トークン('\texttt{b}')を排出する.
    \item 文字'\texttt{c}'を消費し, 文字トークン('\texttt{c}')を排出する.
    \item 文字'\texttt{<}'を消費し, Tag open state に遷移する.
    \item 文字'\texttt{/}'を消費し, End tag open state に遷移する.
    \item 文字'\texttt{a}'を消費し, 名前が空文字である新たな終了タグトークンを作る. Tag name stateに遷移する.
    \item 先ほど消費した文字'\texttt{a}'を再度消費し, 終了タグトークンの名前に'\texttt{a}'を付け足す.
    \item 文字'\texttt{>}'を消費し, 終了タグトークンを排出し, Data stateに遷移する.
    \item EOFトークンを排出する.
\end{enumerate}
動作の結果として,\\
開始タグトークン(名前: ``\texttt{a}'', 属性: [], セルフクローズフラグ: off), 文字トークン('\texttt{b}'), 文字トークン('\texttt{c}'), 終了タグトークン(名前: ``\texttt{a}''), EOFトークン\\
が順に排出される.

\subsection{例2}
%エラーが出る例
入力``\texttt{a<ab}''に対して, HTML5字句解析器は以下ように動作を行う.\\
\begin{enumerate}
    \item 初期状態Data stateにおいて, 文字'\texttt{a}'を消費し, 文字トークン('\texttt{a}')を排出する.
    \item 文字'\texttt{<}'を消費し, Tag open state に遷移する.
    \item 文字'\texttt{a}'を消費し, 名前が空文字である新たな開始タグトークンを作る. Tag name stateに遷移する.
    \item 先ほど消費した文字'\texttt{a}'を再度消費し, 開始タグトークンの名前に'\texttt{a}'を付け足す.
    \item 文字'\texttt{b}'を消費し, 開始タグトークンの名前に'\texttt{b}'を付け足す.
    \item ``eof-in-tag''構文エラーを出す. EOFトークンを排出する.
\end{enumerate}
動作の結果として,\\
``eof-in-tag''構文エラーと, \\
文字トークン('\texttt{a}'), EOFトークンが排出される.
\end{document}