\documentclass[a4paper,twocolumn]{jsarticle}
\usepackage{amsmath,amsthm,txfonts}
%\usepackage[number]{abstract}   %%% abstractをone-column  『1 概要』となる。
\usepackage{abstract}   %%% abstractをone-column
\usepackage{balance}    %%% 最後のページの2段組の高さを揃える。\balanceを入れる。
                        %%% そろえたくないときは、\nobalance
\usepackage{graphicx}   %%% 図を入れる。

\setlength{\columnseprule}{0.3pt}

\title{論文形式の例}
\author{Yoshiki KUMAZAWA}
\date{\today}

\begin{document}

\twocolumn[
\maketitle
\begin{onecolabstract}
2段組、概要は1段、2段組最終ページのコラム長さを揃える。
\vskip 5mm
\end{onecolabstract}
]

\balance
%\nobalance

\section{Lisa}

21日付のニューヨーク

\begin{table*}
\centering
\begin{tabular}{cc|c|c|c} \hline
\multicolumn{2}{c|}{あ} & い & う & え \\\hline
\multicolumn{2}{c|}{か} & き & く & け \\\hline
\end{tabular}
\caption{表}
\end{table*}


中村は梨田監督
\begin{equation}
f(x)=ddd
\end{equation}

近鉄の足高管理部長は20日

\section{test}


大Knuth (1984)を読んで

\begin{figure*}
\centering
\includegraphics[width=3cm,clip]{tiger.eps}
\caption{図}
\end{figure*}

\TeX のことは、Lamport (1994)やKnuth (1984)を読んで

\section{Lisa}

1979年、アップル社
\section{参考文献}

\begingroup
\parindent=0pt   %% default=20pt

\hangindent=20pt %% default=0pt
Leslie Lamport.  \newblock \emph{{\LaTeX:} A Document
    Preparation System}.  \newblock Addison-Wesley, Reading,
  Massachusetts, second edition, 1994, ISBN~0-201-52983-1.
  
\hangindent=20pt %% default=0pt
Donald~E. Knuth.  \newblock \textit{The \TeX{}book,}
  Volume~A of \textit{Computers and Typesetting}, Addison-Wesley,
  Reading, Massachusetts, second edition, 1984, ISBN~0-201-13448-9.

\endgroup

\begin{center}
Yoshiki KUMAZAWA \\
1-1-1 Bamba, Hikone\\
Faculty of Economics,  Shiga University\\
Shiga, JAPAN\\
E-mail: kumazawaアットbiwako.shiga-u.ac.jp
\end{center}

\end{document}